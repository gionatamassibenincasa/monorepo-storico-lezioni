%%%
%%% 1. Definire \dataesame
\def\dataesame{Ancona, 19 gennaio 2022}
%%% 1. Definire \classe
\def\classe{3A SIA}

\documentclass[10pt,a4paper,addpoints%
%,answers
]{exam}
\usepackage[utf8]{inputenc}
\usepackage[T1]{fontenc}
\usepackage[italian]{babel}
\usepackage{amsmath}
\usepackage{amssymb}
\usepackage[left=2cm,top=1.5cm,right=2.5cm,bottom=1.5cm,nohead,nofoot]{geometry}
\usepackage[svgnames]{xcolor}
\usepackage{tikz}
\usetikzlibrary{backgrounds,calc,arrows,patterns}
\usepackage{mathtools}
\usepackage{pdfpages}
\usepackage{multirow}
\usepackage[keeptemps]{pythontex}
\usepackage{hyperref}
\ifx\hypersetup\undefined
  \AtBeginDocument{%
    \hypersetup{unicode=true,
 bookmarks=false,bookmarksnumbered=false,bookmarksopen=false,
 breaklinks=false,pdfborder={0 0 0},backref=false,colorlinks=false,
 pdfstartview=FitH,pdftitle={Compito d'Informatica -- \dataesame},
 pdfauthor={Gionata Massi <gionata.massi@savoiabenincasa.it>},
 pdfsubject={Compito d'Informatica},
 pdfkeywords={Informatica, Metodo numerici}}
  }
\else
  \hypersetup{unicode=true,
 bookmarks=false,bookmarksnumbered=false,bookmarksopen=false,
 breaklinks=false,pdfborder={0 0 0},backref=false,colorlinks=false,pdftitle={Informatica - \dataesame},
 pdfauthor={Gionata Massi <gionata.massi@savoiabenincasa.it>},
 pdfsubject={Compito d'Informatica},
 pdfkeywords={Compito d'Informatica}}
\fi
\usepackage{calc}

\shadedsolutions
\definecolor{SolutionColor}{rgb}{0.8,0.9,1}

\pagestyle{headandfoot}
\firstpageheader{Compito d'Informatica -- \classe}{IIS ``Savoia-Benincasa''}{\dataesame}
\runningheader%
{\makebox[0.49\textwidth]{\textbf{Nome}:~\dotfill}}%
{}%
{\makebox[0.49\textwidth]{\textbf{Cognome}:~\dotfill}}
\firstpagefooter{}{Pagina \thepage\ di \numpages}{}
\runningfooter{Informatica}{Pagina \thepage\ di \numpages}{\dataesame}
\runningfootrule

\addpoints
\qformat{{\bfseries Quesito \thequestion}~(\totalpoints~\points)~\hfill~Punteggio ottenuto: \dots/\totalpoints}
\pointpoints{punto}{punti}
\hqword{Quesito:}
\vqword{Quesito} 
\hpgword{Pg:}
\vpgword{Pagina}
\gradetablestretch{1.5}
\hpword{Max:}
\vpword{Max} 
\hsword{Punti:}
\vsword{Punti}
\htword{Tot.}
\vtword{Voto}
\renewcommand{\solutiontitle}{\noindent\textbf{Svolgimento:}\par\noindent} 

\totalformat{Totale: \totalpoints ~ \points}

\newlength\pageleft
\newlength{\altezzagriglia}

\newcommand{\solutiongridbrk}{
\ifprintanswers {%{\sffamily\bfseries Svolgimento}
}\else{%
  \par %
  \pagebreak[1] %
  \setlength\pageleft{\pagegoal} %
  \addtolength\pageleft{-\pagetotal} %
  \par %
  \tikzset{background grid/.style={very thin,draw=black!25,step=.5cm},background rectangle/.style={draw=black!50!red}} %
  \begin{tikzpicture}[framed,gridded] %
	\color{black!50!red};
    \draw (0,0) (\textwidth-\marginparwidth,\pageleft-4\baselineskip); %
  \end{tikzpicture} %
  \newpage} 
 %\else {}
\fi
}

\newcommand{\solutiongrid}[1]{
\ifprintanswers {%{\sffamily\bfseries Svolgimento}
}\else{%
  \par %
%  \pagebreak[0] %
  \setlength\pageleft{\pagegoal} %   spazio per il testo
  \addtolength\pageleft{-\pagetotal} % - la distanza dall'alto
  \par %
  \tikzset{background grid/.style={very thin,draw=black!25,step=.5cm},background rectangle/.style={draw=black!50}} %
  \setlength{\altezzagriglia}{\minof{#1}{\pageleft-4\baselineskip}}
  \begin{tikzpicture}[framed,gridded] %
    \draw (0,0) (\textwidth-\marginparwidth,\altezzagriglia); %
  \end{tikzpicture} %
  } \fi
}


\newcommand*\circled[1]{\tikz[baseline=(char.base)]{%
            \node[shape=circle,draw,inner sep=2pt] (char) {#1};}}

\begin{document}

\mbox{\vspace{0.5cm}}

\begin{center}
%\fbox{\parbox{0.95\textwidth}{\vspace{0.3in}%
	\makebox[0.49\textwidth]{\textbf{Nome}:~\dotfill} %
	\makebox[0.49\textwidth]{~\textbf{Cognome}:~\dotfill}%
%\vspace{0.2in}}}
\end{center}

\vspace{0.25cm}

\fbox{\parbox{\textwidth}{%
\textbf{Non si pu\`o usare il computer n\'e si possono consultare libri o appunti.}

\textbf{Tempo di consegna: 1 ora.}
}}


\vspace{0.25cm}

\begin{pycode}
from random import randrange
from math import ceil, floor

def gen_erone():
  radicando = randrange(2, 20, 1) * randrange(3, 20, 1) + 1
  #radice = radicando ** 0.5
  #lowerbound = floor(radice) - 1
  #upperbound = ceil(radice) + 1
  return r'$$x = \sqrt{' + str(radicando) + r'}$$' + "\n" + \
        r'\begin{center}' + "\n" + \
        r'\begin{tabular}{|c|p{3.5cm}|p{3.5cm}|}\hline' + "\n" + \
        r'\multicolumn{3}{|c|}{\bf Iterazioni del metodo di Erone}\\\hline' + "\n" + \
        r'\multicolumn{1}{|c|}{$k$}&' + "\n" + \
        r'\multicolumn{1}{|c|}{$\displaystyle \frac{x_k + \frac{' + \
        str(radicando) + \
        r'}{x_k}}{2}$}&' + "\n" + \
        r'\multicolumn{1}{|c|}{$|x_{k} - x_{k-1}|$}\\\hline' + "\n" + \
        r'\multicolumn{1}{|c|}{0}&\multicolumn{1}{|c|}{1}&\multicolumn{1}{|c|}{---}\\\hline' + "\n" + \
        r'1&&\\\hline' + "\n" + \
        r'2&&\\\hline' + "\n" + \
        r'3&&\\\hline' + "\n" + \
        r'4&&\\\hline' + "\n" + \
        r'\end{tabular}' + "\n" + \
        r'\end{center}'

def gen_dicotomico_radice():
  radicando = randrange(2, 20, 1) * randrange(3, 20, 1) + 1
  radice = radicando ** 0.5
  while radice == floor(radice):
    radicando = randrange(2, 20, 1) * randrange(3, 20, 1) + 1
    radice = radicando ** 0.5
  
  lowerbound = floor(radice) - 1
  upperbound = ceil(radice) + 1
  return r'$$f(x) = x^2 - ' + str(radicando) + r',\quad x \in [' + str(lowerbound) + r', ' + str(upperbound) + r']$$' + "\n" + \
         r'\begin{center}' + "\n" + \
         r'\begin{tabular}{|c|*{5}{p{2cm}|}}' + "\n" + \
         r'\hline' + "\n" + \
         r'\multicolumn{6}{|c|}{\bf Iterazioni del metodo dicotomico}\\\hline' + "\n" + \
         r'\multicolumn{1}{|c|}{$k$}&' + "\n" + \
         r'\multicolumn{1}{|c|}{$a_k$}&' + "\n" + \
         r'\multicolumn{1}{|c|}{$b_k$}&' + "\n" + \
         r'\multicolumn{1}{|c|}{$x_k$}&' + "\n" + \
         r'\multicolumn{1}{|c|}{$\mathrm{sgn}(x_k^2 -' + str(radicando) + r')$}&' + "\n" + \
         r'\multicolumn{1}{|c|}{$b_{k} - a_{k}$}\\\hline' + "\n" + \
         r'\multicolumn{1}{|c|}{0}&\multicolumn{1}{|c|}{' + \
         str(lowerbound) + r'}&\multicolumn{1}{|c|}{' + \
         str(upperbound) + r'}&&&\multicolumn{1}{|c|}{---}\\\hline' + "\n" + \
         r'1&&&&&\\\hline' + "\n" + \
         r'2&&&&&\\\hline' + "\n" + \
         r'3&&&&&\\\hline' + "\n" + \
         r'4&&&&&\\\hline' + "\n" + \
         r'\end{tabular}' + "\n" + \
         r'\end{center}' + "\n"

\end{pycode}
\begin{questions}
 \begin{question}
  Si determini un'approssimazione della radice quadrata del radicando eseguendo 5 iterazioni
  del metodo di Erone.
  Si usi la stima iniziale $x_0 = 1$ e si registrino i valori calcolati ad ogni iterazione del metodo
   nelle tabelle seguenti trascrivenddo il valore approssimato alle prime 5 cifre significative.
  \begin{parts}
    \part[2]\py{gen_erone()}
    \part[2]\py{gen_erone()}
  \end{parts}
 \end{question}

 \begin{question}
  Si determini un'approssimazione di $\alpha | f(\alpha) = 0$, dove $f(x) = x^2 - r$, eseguendo 5 iterazioni del metodo dicotomico.
  Si registrino i valori calcolati ad ogni iterazione del metodo nella tabella seguente approssimando con
  5 cifre significative mantenedo gli invarianti ${\mathrm sgn} (a_k^2 - r) \leq 0$ e
  ${\mathrm sgn} (b_k^2 - r) \geq 0$.
  \begin{parts}
    \part[3]\py{gen_dicotomico_radice()}
    \part[3]\py{gen_dicotomico_radice()}
  \end{parts}
 \end{question}

\end{questions}
\flushright{(Totale compito: \numpoints ~punti)}

%\vfill

%\begin{center}
%\gradetable[v][questions]
%\end{center}

%\vfill

\end{document}

r'\multicolumn{1}{|c|}{$\mathrm{sgn}(a_k^2 -' + radicando + r')$}&' + "\n" + \
r'\multicolumn{1}{|c|}{$\mathrm{sgn}(b_k^2 -' + radicando + r')$}&' + "\n" + \
