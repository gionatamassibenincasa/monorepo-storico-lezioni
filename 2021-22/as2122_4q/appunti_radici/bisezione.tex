\section{Metodo di bisezione}

Sia $f (x)$ una funzione continua sull'intervallo limitato e chiuso $[a, b]$ con $f (a) \cdot f (b) < 0$.

L'algoritmo genera una successione di intervalli
$[a_k , b_k]$ con $f (a_k ) \cdot f (b_k ) < 0$ e con $[a_k , b_k ] \subset [a_{k-1}$ , $b_{k-1} ]$ e $|b_k - a_k | = \frac{1}{2}|b_{k-1} - a_{k-1} |$. 

Date due tolleranze $\epsilon_1$ , $\epsilon_2$ , l’algoritmo si arresta quando:

\begin{itemize}
    \item $|b_k - a_k | \leq \epsilon_1$, oppure
    \item $|f (\frac {a_k +b_k}{2} )| \leq \epsilon_2$, o infine
    \item $k > \text{nmax}$, ove nmax è un numero massimo di iterazioni fissato.
\end{itemize}

Per alleggerire la notazione usiamo $s_{a_k}$ per indicare $\mathrm{segno} \left(f\left(a_k\right)\right)$,
$s_{b_k}$ per $\mathrm{segno}(f(b_k))$ e
$s_k$ per $\mathrm{segno}(f (\frac {a_k +b_k}{2}))$, dove

\begin{equation}
    \mathrm{segno}(x) =
    \begin{cases}
        -1 & \text{se } x < 0, \\
         0 & \text{{se }} x = 0, \\
        +1 & \text{{se }} x > 0
    \end{cases}
\end{equation}

Per il calcolo di $\frac{a_k+b_k}{2}$ in virgola mobile si deve usare la formula: $a_k + (b_k - a_k) / 2$ in modo da ridurre gli errori di troncamento.

\subsection{Convergenza}

Il metodo è sempre convergente e si può calcolare l'ordine considerando l'errore
come l'ampiezza dell'intervallo di incertezza: $\varepsilon_k = \frac{|b - a|}{2^k}$.
Il rapporto tra due errori successivi per $p=1$ vale

$$\displaystyle \frac{\varepsilon_{k+1}}{\varepsilon_{k}^p}
=\frac{
    \frac{|b - a|}{2^{k+1}}
  }{
      \frac{|b - a|}{2^k}
  }
  = \frac{1}{2}$$

L'ordine di convergenza è uno con costante d'errore vale un mezzo.

\subsection{Codifica in JavaScript}

Si veda il listato \ref{lst:bisezione}.

\begin{lstfloat}
    \lstinputlisting{lst/bisezione.js}
    \caption{Descrizione in JavaScript del metodo di Bisezione}
    \label{lst:bisezione}
\end{lstfloat}

\subsection{Esempi}

Si vedano le tabb.~\ref{tbl:tab_bis_sqrt_6}, \ref{tbl:tab_bis_sin} e \ref{tbl:tab_bis_exp_mx} e le figg. \ref{fig:bis_sqrt_6} e \ref{fig:subplots_bis_sqrt_6}.

\begin{table}[!htbp]
    \begin{center}
\pgfplotstabletypeset[
	col sep=tab,
    every head row/.style={before row=\toprule,after row=\midrule},	% style the first row
	every last row/.style={after row=\bottomrule},	% style the last row
    every column/.style={dec sep align,precision=10}
]{tbl/tab_bis_sqrt_6.dat}
\end{center}        
\caption[]{Metodo dicotomico applicato a $x^2 -6$ nell'intervallo $[0, 6]$ con nmax = 10}
\label{tbl:tab_bis_sqrt_6}
\end{table}

\begin{figure}[!htbp]
    \centering
    \includestandalone{img/iter_bis_sqrt_6}
    \caption{Successione delle soluzioni del metodo dicotomico applicato a $x^2 -6$ nell'intervallo $[0, 6]$}
    \label{fig:bis_sqrt_6}
\end{figure}

\begin{figure}[!htbp]
    \centering
    \includestandalone{img/iter_bis_sqrt_6_stacked}
    \caption{Passi del metodo dicotomico applicato a $x^2 -6$ nell'intervallo $[0, 6]$}
    \label{fig:subplots_bis_sqrt_6}
\end{figure}


\begin{table}[!htbp]
    \begin{center}
        \pgfplotstabletypeset[
            col sep=tab,
            every head row/.style={before row=\toprule,after row=\midrule},	% style the first row
            every last row/.style={after row=\bottomrule},	% style the last row
            every column/.style={dec sep align,precision=10}
        ]{tbl/tab_bis_sin.dat}
    \end{center}        
    \caption{Metodo dicotomico applicato a $sin(x)$ nell'intervallo $[3, 3.2]$ con nmax = 10}
    \label{tbl:tab_bis_sin}
\end{table}

\begin{table}[!htbp]
    \begin{center}
        \pgfplotstabletypeset[
            col sep=tab,
            every head row/.style={before row=\toprule,after row=\midrule},	% style the first row
            every last row/.style={after row=\bottomrule},	% style the last row
            every column/.style={dec sep align,precision=10}
            %columns/.style={sci,sci subscript,sci zerofill,dec sep align}
            %every first column/.style={column type/.add={|}{}},	% style the first column
            %every last column/.style={column type/.add={}{|}},	% style the last column
            %columns/C/.style = {column type/.add={|}{|}}	% style the designated column
        ]{tbl/tab_bis_exp_mx.dat}
    \end{center}        
    \caption{Metodo dicotomico applicato a $e^{e^{-x}}-x$ nell'intervallo $[0, 1]$ con nmax = 10}
    \label{tbl:tab_bis_exp_mx}
\end{table}
