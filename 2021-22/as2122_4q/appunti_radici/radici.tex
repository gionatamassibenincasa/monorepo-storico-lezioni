\documentclass[12pt]{article} 
\usepackage[utf8]{inputenc}
\usepackage[T1]{fontenc}
\usepackage{geometry}
\geometry{a4paper}
\geometry{top=0.5in, bottom=0.5in, left=.7in, right=.7in}
%\usepackage{blindtext}
\usepackage{hyperref}
\usepackage{bookmark}
\usepackage{booktabs} % for much better looking tables
\usepackage{array} % for better arrays (eg matrices) in maths
%\usepackage{paralist} % very flexible & customisable lists (eg. enumerate/itemize, etc.)
%\usepackage{verbatim} % adds environment for commenting out blocks of text & for better verbatim
\usepackage{subfig} % make it possible to include more than one captioned figure/table in a single float
\usepackage{placeins}
% These packages are all incorporated in the memoir class to one degree or another...
\usepackage[italian]{babel}

%%% HEADERS & FOOTERS
\usepackage{fancyhdr} % This should be set AFTER setting up the page geometry
\pagestyle{fancy} % options: empty , plain , fancy
\renewcommand{\headrulewidth}{0pt} % customise the layout...
\lhead{}\chead{}\rhead{}
\lfoot{}\cfoot{\thepage}\rfoot{}

%%% SECTION TITLE APPEARANCE
\usepackage{sectsty}
\allsectionsfont{\sffamily\mdseries\upshape} % (See the fntguide.pdf for font help)
% (This matches ConTeXt defaults)

%%% ToC (table of contents) APPEARANCE
\usepackage[nottoc,notlof,notlot]{tocbibind} % Put the bibliography in the ToC
\usepackage[titles,subfigure]{tocloft} % Alter the style of the Table of Contents
\renewcommand{\cftsecfont}{\rmfamily\mdseries\upshape}
\renewcommand{\cftsecpagefont}{\rmfamily\mdseries\upshape} % No bold!

\usepackage{amsmath}
\usepackage{amsfonts}
\usepackage{amsthm}
\usepackage{systeme}

\theoremstyle{plain}% default
\newtheorem{thm}{Teorema}[section]
\newtheorem{lem}[thm]{Lemma}
\newtheorem{prop}[thm]{Proposizione}
\newtheorem*{cor}{Corollario}
\newtheorem*{KL}{Lemma di Klein}
\theoremstyle{definition}
\newtheorem{defn}{Definizione}[section]
\newtheorem{conj}{Congettura}[section]
\newtheorem{exmp}{Esempio}[section]
\theoremstyle{remark}
\newtheorem*{comm}{Commento}
\newtheorem*{note}{Nota}
\newtheorem{caso}{Caso}
\usepackage[mode=buildnew]{standalone}% requires --shell-escape
\usepackage{tikz}
\usepackage{pgfplots}
\pgfplotsset{width=10cm,compat=newest}
\usepackage{pgfplotstable}

\usepackage{multirow}
% \usepackage[keeptemps]{pythontex}
\usepackage{float}
\usepackage{listings}
\lstdefinelanguage{JavaScript}{
  keywords={typeof, new, true, false, catch, function, return, null, catch, switch, var, if, in, while, do, else, case, break},
  keywordstyle=\color{blue}\bfseries,
  ndkeywords={class, export, boolean, throw, implements, import, this},
  ndkeywordstyle=\color{darkgray}\bfseries,
  identifierstyle=\color{black},
  sensitive=false,
  comment=[l]{//},
  morecomment=[s]{/*}{*/},
  commentstyle=\color{purple}\ttfamily,
  stringstyle=\color{red}\ttfamily,
  morestring=[b]',
  morestring=[b]"
}

\lstset{
   language=JavaScript,
   backgroundcolor=\color{lightgray},
   extendedchars=true,
   basicstyle=\footnotesize\ttfamily,
   showstringspaces=false,
   showspaces=false,
   numbers=left,
   numberstyle=\footnotesize,
   numbersep=9pt,
   tabsize=2,
   breaklines=true,
   showtabs=false,
   captionpos=b
}

\newfloat{lstfloat}{htbp}{lop}
\floatname{lstfloat}{Codice sorgente}
\def\lstfloatautorefname{Listato} % needed for hyperref/auroref


%%% The "real" document content comes below...

\title{Alcuni metodi iterativi per la ricerca di radici di funzioni}
\author{Gionata Massi}
\date{} % Activate to display a given date or no date (if empty),
         % otherwise the current date is printed 

\begin{document}
\maketitle

\thispagestyle{empty}%\frenchspacing

\tableofcontents

\thispagestyle{empty}
\listoffigures
\listoftables
\newpage
\pagenumbering{arabic}

\section{Il problema}

Data una funzione $f : \mathbb{R} \to \mathbb{R}$, determinare un valore reale $\alpha$ tale che $f(\alpha) = 0$.

Usualmente consideriamo funzioni continue in $\mathbb{R}$ o almeno in un intevallo $[a, b] \subseteq \mathbb{R}$ chiuso e limitato in cui ricercare una radice.

\section{Esistenza delle radici}

Non tutte le funzioni ammettono radici, ad esempio $x \mapsto k$ e $x \mapsto (x + k)^2$, dove $k \neq 0$ (es: fig.~\ref{fig:no_zeri}).

\begin{figure}[!htbp]
    \centering
    \includestandalone{img/no_zeri}
    \caption{Funzioni che non intersecano l'asse $y = 0$}
    \label{fig:no_zeri}
\end{figure}
    
Altre funzioni hanno radici nei punti di massimo o di minimo locale (es: fig. \ref{fig:zero_estremante})
e questo può rendere difficile identificare già la sola esistenza degli zeri. Se un metodo può essere eseguito,
esso sarà poco efficiente nel trovare la soluzione vicino ad un estremante.

\begin{figure}[!htbp]
    \centering
    \includestandalone{img/zero_estremanti}
    \caption{Funzioni che intersecano l'asse $y = 0$ in un estremante}
    \label{fig:zero_estremante}
\end{figure}

\subsection{Teorema degli zeri}

Per essere sicuri che una funzione ammetta almeno una radice richiediamo che la funzione assuma valori positivi e negativi in un certo intervallo e che sia continua.

\begin{thm}[Bolzano]
Se $f (x)$ è una funzione continua sull'intervallo limitato e chiuso $[a, b]$ e $f (a) \cdot f (b) < 0$, allora esiste almeno una radice di $f (x)$ nell'intervallo $(a, b)$.
\end{thm}

Se le ipotesi del teorema sono vere può esistere una sola radice oppure ce ne possono essere in numero finito o anche infinite (fig. \ref{fig:ipotesibolzano}).

\begin{figure}[!htbp]
    \centering
    \includestandalone{img/segni_discordi}
    \caption{Funzioni che assumono valori opposti agli estremi -1, 1}
    \label{fig:ipotesibolzano}
\end{figure}

Un metodo di ricerca delle radici, se convergente, restituirà una sola delle radici.
Si intuisce che maggiore è la pendenza della funzione in un intorno della radice, più è facile discriminare la radice. Se invece la pendenza è nulla o quasi, allora il problema si dice mal condizionato.

Esistono funzioni continue in cui è difficile anche solo enumerare gli zeri.

Si consideri nell'intervallo [-1, 1] la seguente funzione:

\begin{equation*}
    f(x) =
    \begin{cases}
        x \sin(\frac{1}{x}) & x \neq 0\\
        0                   & x = 0
    \end{cases}
\end{equation*}

Il grafico in fig.~\ref{fig:tantizeri} è costruito campionando l'intervallo [-1, 1] in 800 punti e
valutando in ognuno di questi punti, con la massima precisione possibile in virgola mobile, il
valore della funzione.
Anche aumentando il numero di punti il grafico ottenuto al calcolare sarà sempre impreciso.

\begin{figure}[!htbp]
    \centering
    \includestandalone{img/tanti_zeri}
    \caption{Funzioni che assumono valori opposti agli estremi -1, 1 e hanno infiniti zeri}
    \label{fig:tantizeri}
\end{figure}

\section{Metodi numerici}

\subsection{Metodi diretti e metodi iterativi}

I \textbf{metodi diretti} sono algoritmi che, in assenza di errori di arrotondamento, forniscono la soluzione in un numero finito di operazioni.

I \textbf{metodi iterativi} sono algoritmi nei quali la soluzione è ottenuta come limite di una successione di soluzioni. Nella risposta fornita da un metodo iterativo è, quindi, presente usualmente un errore di troncamento.

\subsection{Ordine di convergenza}

Sia $x_k, k = 0, 1,\ldots$ una successione convergente al valore $\alpha$.

\begin{defn}[Errore all'iterazione $k$-esima]
L'errore per elemento della successione in posizione $k$ è il valore non negativo
$\varepsilon_k = |x_k - \alpha|$.
\end{defn}

\begin{defn}[Ordine di convergenza]
Se esistono un numero reale $p \geq 1$ e una costante reale positiva $C$
tale che
$\lim_{k\to+\infty} \frac{\varepsilon_{k+1}}{\varepsilon_{k}^p}=C$
allora la successione $x_k, k = 0, 1,\ldots$ ha ordine di convergenza $p$
e costante d'errore $C$.
\end{defn}

Si noti che nella definizione di ordine di convergenza, la costante $C$
è positiva, ossia strettamente maggiore di zero.

\subsection{Criteri di arresto}

I metodi iterativi sviluppano una successione di valori che deve essere troncata
per produrre un risultato.
Si posso usare vari criteri per arrestare un metodo numerico in base a vincoli
sull'approssimazione della soluzione cercata e sulle risorse di calcolo
(essenzialmente il tempo di calcolo) da impiegare.
I criteri di arresto comunemente adottati sono i seguenti:

\begin{description}
\item[Tolleranza sull'approssimazione della funzione]$|f(x_k)| < \tau_y$.
\item[Tolleranza assoluta sull'approssimazione di $\alpha$] $|x_{k + 1} - x_{k}| < \tau_x$.
\item[Tolleranza relativa sull'approssimazione di $\alpha$] $\frac{|x_{k + 1} - x_{k}|}{|x_{k+1}|} < \tau_r$.
\item[Numero di iterazioni] $k < k_M$. 
\end{description}

% Si veda Comincioli2004.

\section{Metodo di bisezione}

Sia $f (x)$ una funzione continua sull'intervallo limitato e chiuso $[a, b]$ con $f (a) \cdot f (b) < 0$.

L'algoritmo genera una successione di intervalli
$[a_k , b_k]$ con $f (a_k ) \cdot f (b_k ) < 0$ e con $[a_k , b_k ] \subset [a_{k-1}$ , $b_{k-1} ]$ e $|b_k - a_k | = \frac{1}{2}|b_{k-1} - a_{k-1} |$. 

Date due tolleranze $\epsilon_1$ , $\epsilon_2$ , l’algoritmo si arresta quando:

\begin{itemize}
    \item $|b_k - a_k | \leq \epsilon_1$, oppure
    \item $|f (\frac {a_k +b_k}{2} )| \leq \epsilon_2$, o infine
    \item $k > \text{nmax}$, ove nmax è un numero massimo di iterazioni fissato.
\end{itemize}

Per alleggerire la notazione usiamo $s_{a_k}$ per indicare $\mathrm{segno} \left(f\left(a_k\right)\right)$,
$s_{b_k}$ per $\mathrm{segno}(f(b_k))$ e
$s_k$ per $\mathrm{segno}(f (\frac {a_k +b_k}{2}))$, dove

\begin{equation}
    \mathrm{segno}(x) =
    \begin{cases}
        -1 & \text{se } x < 0, \\
         0 & \text{{se }} x = 0, \\
        +1 & \text{{se }} x > 0
    \end{cases}
\end{equation}

Per il calcolo di $\frac{a_k+b_k}{2}$ in virgola mobile si deve usare la formula: $a_k + (b_k - a_k) / 2$ in modo da ridurre gli errori di troncamento.

\subsection{Convergenza}

Il metodo è sempre convergente e si può calcolare l'ordine considerando l'errore
come l'ampiezza dell'intervallo di incertezza: $\varepsilon_k = \frac{|b - a|}{2^k}$.
Il rapporto tra due errori successivi per $p=1$ vale

$$\displaystyle \frac{\varepsilon_{k+1}}{\varepsilon_{k}^p}
=\frac{
    \frac{|b - a|}{2^{k+1}}
  }{
      \frac{|b - a|}{2^k}
  }
  = \frac{1}{2}$$

L'ordine di convergenza è uno con costante d'errore vale un mezzo.

\subsection{Codifica in JavaScript}

Si veda il listato \ref{lst:bisezione}.

\begin{lstfloat}
    \lstinputlisting{lst/bisezione.js}
    \caption{Descrizione in JavaScript del metodo di Bisezione}
    \label{lst:bisezione}
\end{lstfloat}

\subsection{Esempi}

Si vedano le tabb.~\ref{tbl:tab_bis_sqrt_6}, \ref{tbl:tab_bis_sin} e \ref{tbl:tab_bis_exp_mx} e le figg. \ref{fig:bis_sqrt_6} e \ref{fig:subplots_bis_sqrt_6}.

\begin{table}[!htbp]
    \begin{center}
\pgfplotstabletypeset[
	col sep=tab,
    every head row/.style={before row=\toprule,after row=\midrule},	% style the first row
	every last row/.style={after row=\bottomrule},	% style the last row
    every column/.style={dec sep align,precision=10}
]{tbl/tab_bis_sqrt_6.dat}
\end{center}        
\caption[]{Metodo dicotomico applicato a $x^2 -6$ nell'intervallo $[0, 6]$ con nmax = 10}
\label{tbl:tab_bis_sqrt_6}
\end{table}

\begin{figure}[!htbp]
    \centering
    \includestandalone{img/iter_bis_sqrt_6}
    \caption{Successione delle soluzioni del metodo dicotomico applicato a $x^2 -6$ nell'intervallo $[0, 6]$}
    \label{fig:bis_sqrt_6}
\end{figure}

\begin{figure}[!htbp]
    \centering
    \includestandalone{img/iter_bis_sqrt_6_stacked}
    \caption{Passi del metodo dicotomico applicato a $x^2 -6$ nell'intervallo $[0, 6]$}
    \label{fig:subplots_bis_sqrt_6}
\end{figure}


\begin{table}[!htbp]
    \begin{center}
        \pgfplotstabletypeset[
            col sep=tab,
            every head row/.style={before row=\toprule,after row=\midrule},	% style the first row
            every last row/.style={after row=\bottomrule},	% style the last row
            every column/.style={dec sep align,precision=10}
        ]{tbl/tab_bis_sin.dat}
    \end{center}        
    \caption{Metodo dicotomico applicato a $sin(x)$ nell'intervallo $[3, 3.2]$ con nmax = 10}
    \label{tbl:tab_bis_sin}
\end{table}

\begin{table}[!htbp]
    \begin{center}
        \pgfplotstabletypeset[
            col sep=tab,
            every head row/.style={before row=\toprule,after row=\midrule},	% style the first row
            every last row/.style={after row=\bottomrule},	% style the last row
            every column/.style={dec sep align,precision=10}
            %columns/.style={sci,sci subscript,sci zerofill,dec sep align}
            %every first column/.style={column type/.add={|}{}},	% style the first column
            %every last column/.style={column type/.add={}{|}},	% style the last column
            %columns/C/.style = {column type/.add={|}{|}}	% style the designated column
        ]{tbl/tab_bis_exp_mx.dat}
    \end{center}        
    \caption{Metodo dicotomico applicato a $e^{e^{-x}}-x$ nell'intervallo $[0, 1]$ con nmax = 10}
    \label{tbl:tab_bis_exp_mx}
\end{table}


\section{Iterazioni di punto fisso}

In generale si può costruire un metodo iterativo cercando un punto fisso di una funzione $\Phi(x)$,
costruita in moda che si annulli ne punto desiderato, un valore $\bar{x}$ tale che $\Phi(\bar{x}) = \bar{x}$.

Il punto fisso è calcolato tramite l'applicazione ripetuta della regola di ricorrenza:

$$x_{k+1} = \Phi(x_k)$$

\subsection{Approssimazioni con rette}

Una retta è definita da una funzione del tipo $r(x) = m x + q$.
Se imponiamo il passaggio per il punto $(x_k, f(x_k))$ della funzione di cui cerchiamo una radice, il fascio di rette sarà:

$$r(x) - f(x_k) = m (x - x_k)$$

Possiamo generare delle iterazioni andando a fissare il coefficiente angolare ad ogni iterazione e determinando l'intersezione della retta con l'asse delle ascisse.

$$r(x_{k+1}) - f(x_k) = m_k (x_{k+1} - x_k)$$

Risolvendo per $$r(x_{k+1}) = 0$$ si ha

\begin{equation}
    x_{k+1} = x_k - \frac{f(x_k)}{m_k}
\end{equation}

\begin{figure}[!htbp]
    \begin{center}
        \includestandalone{img/approx_rette}
        \caption{Approssimazione di una funzione con una retta passante per ($x_k$, $f(x_k)$)}
        \label{fig:retta_approx}
    \end{center}
\end{figure}

% \subsection{Metodo delle corde}

Si considerino due punti $a=x_0$ e $b=x_1$ tali da soddisfare le ipotesi del teorema di Bolzano.
È possibile costruire una successione che per ogni $k \geq 0$  il punto $x_{k+1}$ sia lo zero della retta passante per il punto
($x_{k}$, $f(x_{k})$) e di coefficiente angolare

$$\displaystyle m_k = \frac{f(a)  - f(x_k)}{a - x_k}.$$

L'iterata ha equazione:

$$x_{k+1} = x_k - f(x_k) \frac{a  - x_k)}{f(a) - f(x_k)}.$$

\subsubsection{Convergenza}

Il metodo non necessariamente converge.
Può oscillare (vedi tab.~\ref{tbl:tab_cor_osc}) o divergere.


\subsubsection{Codifica in JavaScript}

Si veda il listato \ref{lst:corde}.

\begin{lstfloat}
    \lstinputlisting{lst/corde.js}
    \caption{Descrizione in JavaScript del metodo delle corde}
    \label{lst:corde}
\end{lstfloat}

\subsubsection{Esempi}

Si vedano le tabb.~\ref{tbl:tab_cor_sqrt_6}, \ref{tbl:tab_cor_sin} e \ref{tbl:tab_cor_exp_mx} e la fig. \ref{fig:cor_sqrt_6}.

\begin{table}
    \begin{center}
\pgfplotstabletypeset[
	col sep=tab,
    every head row/.style={before row=\toprule,after row=\midrule},	% style the first row
	every last row/.style={after row=\bottomrule},	% style the last row
    every column/.style={dec sep align,precision=10}
]{tbl/tab_bis_sqrt_6.dat}
\end{center}        
\caption[]{Metodo delle corde applicato a $x^2 -6$ nell'intervallo $[0, 6]$ con nmax = 10}
\label{tbl:tab_cor_sqrt_6}
\end{table}

\begin{figure}[ht]
    \centering
    \includestandalone{img/iter_cor_sqrt_6}
    \caption{Successione delle soluzioni del metodo delle corde applicato a $x^2 -6$ nell'intervallo $[0, 6]$}
    \label{fig:cor_sqrt_6}
\end{figure}

\begin{table}
    \begin{center}
        \pgfplotstabletypeset[
            col sep=tab,
            every head row/.style={before row=\toprule,after row=\midrule},	% style the first row
            every last row/.style={after row=\bottomrule},	% style the last row
            every column/.style={dec sep align,precision=10}
        ]{tbl/tab_cor_sin.dat}
    \end{center}        
    \caption[]{Metodo delle corde applicato a $sin(x)$ nell'intervallo $[3, 3.2]$ con nmax = 10}
    \label{tbl:tab_cor_sin}
\end{table}

\begin{table}
    \begin{center}
        \pgfplotstabletypeset[
            col sep=tab,
            every head row/.style={before row=\toprule,after row=\midrule},	% style the first row
            every last row/.style={after row=\bottomrule},	% style the last row
            every column/.style={dec sep align,precision=10}
            %columns/.style={sci,sci subscript,sci zerofill,dec sep align}
            %every first column/.style={column type/.add={|}{}},	% style the first column
            %every last column/.style={column type/.add={}{|}},	% style the last column
            %columns/C/.style = {column type/.add={|}{|}}	% style the designated column
        ]{tbl/tab_cor_exp_mx.dat}
    \end{center}        
    \caption[]{Metodo delle corde applicato a $e^{e^{-x}}-x$ nell'intervallo $[0, 1]$ con nmax = 10}
    \label{tbl:tab_cor_exp_mx}
\end{table}

\begin{table}
    \begin{center}
\pgfplotstabletypeset[
	col sep=tab,
    every head row/.style={before row=\toprule,after row=\midrule},	% style the first row
	every last row/.style={after row=\bottomrule},	% style the last row
    every column/.style={dec sep align,precision=10}
]{tbl/tab_cor_osc.dat}
\end{center}        
\caption[]{Metodo delle corde applicato a $x^2 -1$ nell'intervallo $[0, 2]$ con nmax = 10}
\label{tbl:tab_cor_osc}
\end{table}


% \subsection{Metodo delle secanti}

Dati due punti iniziali $x_0$ e $x_1$, si considera la secante passante per i due punti dati.

In generale il coefficiente angolare $m_k$ è calcolato come:

$$m_k = \frac{f(x_k) - f(x_{k-1})}{x_k - x_{k-1}}.$$

L'iterata è

$$x_{k+1} = x_k - f(x_k) \cdot \frac{x_k - x_{k-1}}{f(x_k) - f(x_{k-1})}.$$

\subsubsection{Convergenza}

Il metodo non necessariamente converge.
Può oscillare o divergere.

\subsubsection{Codifica in JavaScript}

\begin{lstfloat}
    \lstinputlisting{lst/secanti.js}
    \caption{Descrizione in JavaScript del metodo delle secanti}
    \label{lst:secanti}
\end{lstfloat}

\subsubsection{Esempi}

Si vedano gli esempi nelle tabb.~\ref{tbl:tab_sec_sqrt_6}, \ref{tbl:tab_sec_sin} e \ref{tbl:tab_sec_exp_mx} e la fig.~\ref{fig:sec_sqrt_6}.


\begin{table}
    \begin{center}
\pgfplotstabletypeset[
	col sep=tab,
    every head row/.style={before row=\toprule,after row=\midrule},	% style the first row
	every last row/.style={after row=\bottomrule},	% style the last row
    every column/.style={dec sep align,precision=10}
]{tbl/tab_bis_sqrt_6.dat}
\end{center}        
\caption[]{Metodo delle secanti applicato a $x^2 -6$ nell'intervallo $[0, 6]$ con nmax = 10}
\label{tbl:tab_sec_sqrt_6}
\end{table}

\begin{figure}[ht]
    \centering
    \includestandalone{img/iter_cor_sqrt_6}
    \caption{Successione delle soluzioni del metodo delle secanti applicato a $x^2 -6$ nell'intervallo $[0, 6]$}
    \label{fig:sec_sqrt_6}
\end{figure}

\begin{table}
    \begin{center}
        \pgfplotstabletypeset[
            col sep=tab,
            every head row/.style={before row=\toprule,after row=\midrule},	% style the first row
            every last row/.style={after row=\bottomrule},	% style the last row
            every column/.style={dec sep align,precision=10}
        ]{tbl/tab_cor_sin.dat}
    \end{center}        
    \caption[]{Metodo delle secanti applicato a $sin(x)$ nell'intervallo $[3, 3.2]$ con nmax = 10}
    \label{tbl:tab_sec_sin}
\end{table}

\begin{table}
    \begin{center}
        \pgfplotstabletypeset[
            col sep=tab,
            every head row/.style={before row=\toprule,after row=\midrule},	% style the first row
            every last row/.style={after row=\bottomrule},	% style the last row
            every column/.style={dec sep align,precision=10}
            %columns/.style={sci,sci subscript,sci zerofill,dec sep align}
            %every first column/.style={column type/.add={|}{}},	% style the first column
            %every last column/.style={column type/.add={}{|}},	% style the last column
            %columns/C/.style = {column type/.add={|}{|}}	% style the designated column
        ]{tbl/tab_cor_exp_mx.dat}
    \end{center}        
    \caption[]{Metodo delle secanti applicato a $e^{e^{-x}}-x$ nell'intervallo $[0, 1]$ con nmax = 10}
    \label{tbl:tab_sec_exp_mx}
\end{table}


\subsection{Metodo delle tangenti}

Si approssima la funzione $f(x)$ con la retta $r(x) - f(x_k) = f'(x_k) (x - x_k)$ tangente ad essa in $(x_k, f(x_k))$.

L'iterata assume la forma:

$$x_{k+1} = x_k - \frac{f(x_k)}{f'{x_k}}.$$

\subsubsection{Convergenza}

Il metodo non necessariamente converge.
Può oscillare o divergere.

\subsubsection{Codifica in JavaScript}


Si vedanil listato \ref{lst:tangenti}.

\begin{lstfloat}
    \lstinputlisting{lst/tangenti.js}
    \caption{Descrizione in JavaScript del metodo delle tangenti}
    \label{lst:tangenti}
\end{lstfloat}

\subsubsection{Esempi}

Si vedano le tabb.~\ref{tbl:tab_tan_sqrt_6}, \ref{tbl:tab_tan_sin} e \ref{tbl:tab_tan_exp_mx} e la fig. \ref{fig:tan_sqrt_6}.

\begin{table}
    \begin{center}
\pgfplotstabletypeset[
	col sep=tab,
    every head row/.style={before row=\toprule,after row=\midrule},	% style the first row
	every last row/.style={after row=\bottomrule},	% style the last row
    every column/.style={dec sep align,precision=10}
	%every first column/.style={column type/.add={|}{}},	% style the first column
	%every last column/.style={column type/.add={}{|}},	% style the last column
	%columns/C/.style = {column type/.add={|}{|}}	% style the designated column
]{tbl/tab_tan_sqrt_6.dat}
\end{center}        
\caption[]{Metodo delle tangenti applicato a $x^2 -6$ con stima iniziale 3 e nmax = 10}
\end{table}

\begin{figure}[ht]
    \centering
    \includestandalone{img/iter_tan_sqrt_6}
    \caption{Successione delle soluzioni del metodo delle tangenti applicato a $x^2 -6$ nell'intervallo $[0, 6]$}
    \label{fig:tan_sqrt_6}
\end{figure}

\begin{table}
    \begin{center}
        \pgfplotstabletypeset[
            col sep=tab,
            every head row/.style={before row=\toprule,after row=\midrule},	% style the first row
            every last row/.style={after row=\bottomrule},	% style the last row
            every column/.style={dec sep align,precision=10}
            %every first column/.style={column type/.add={|}{}},	% style the first column
            %every last column/.style={column type/.add={}{|}},	% style the last column
            %columns/C/.style = {column type/.add={|}{|}}	% style the designated column
        ]{tbl/tab_tan_sin.dat}
    \end{center}        
    \caption[]{Metodo delle tangenti applicato a $sin(x)$ con stima iniziale 3,1 e nmax = 10}
\end{table}

\begin{table}
    \begin{center}
        \pgfplotstabletypeset[
            col sep=tab,
            every head row/.style={before row=\toprule,after row=\midrule},	% style the first row
            every last row/.style={after row=\bottomrule},	% style the last row
            every column/.style={dec sep align,precision=10}
            %columns/.style={sci,sci subscript,sci zerofill,dec sep align}
            %every first column/.style={column type/.add={|}{}},	% style the first column
            %every last column/.style={column type/.add={}{|}},	% style the last column
            %columns/C/.style = {column type/.add={|}{|}}	% style the designated column
        ]{tbl/tab_tan_exp_mx.dat}
    \end{center}        
    \caption[]{Metodo delle tangenti applicato a $e^{e^{-x}}-x$ con stima iniziale 0,5 e nmax = 10}
\end{table}


\subsubsection{Esempio: algoritmo del reciproco}

Si vuole cercare il reciproco del valore $\nu$ come $\alpha = \frac{1}{\nu}$ con il metodo delle tangenti.

Per prima cosa occorre trasformare il problema con una funzione che si annulla in $\frac{1}{\nu}$.

Scegliamo $$f(x) = \nu - \frac{1}{x}$$ che applicata al reciproco di $\nu$ produce
$f(\frac{1}{\nu}) = \nu - \frac{1}{\frac{1}{\nu}} = \nu - \nu = 0$. 

La derivata prima assume la forma $f'(x) = \frac{1}{x^2}$ e

$$\frac{f(x)}{f'(x)} = \frac{\nu - \frac{1}{x}}{\frac{1}{x^2}} = \nu x^2 - x.$$

L'iterata del metodo delle tangenti è:

$$x_{k+1} = x_k - (\nu x_k^2 - x_k) = 2 x_k - \nu x_k^2  = x_k \cdot (2 - \nu \cdot x_k).$$

Si noti che per calcolare il reciproco di un numero sono sufficienti le operazioni di moltiplicazione e sottrazione.
Per la moltiplicazione occorrono le operazioni primitive di scorrimento e addizione e per la sottrazione quelle di negazione bit a bit e di addizione (basterebbe il solo incremento unitario).

Nota: queste proprietà permettono di realizzare le CPU senza l'operazione di divisione.



% \input{altri_metodi}

\end{document}
