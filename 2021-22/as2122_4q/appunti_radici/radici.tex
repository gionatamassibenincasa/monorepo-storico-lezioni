\documentclass[12pt]{article} 
\usepackage[utf8]{inputenc}
\usepackage{geometry}
\geometry{a4paper}
\geometry{top=0.5in, bottom=0.5in, left=.7in, right=.7in}
%\usepackage{blindtext}
\usepackage{hyperref}
\usepackage{bookmark}
\usepackage{booktabs} % for much better looking tables
\usepackage{array} % for better arrays (eg matrices) in maths
%\usepackage{paralist} % very flexible & customisable lists (eg. enumerate/itemize, etc.)
%\usepackage{verbatim} % adds environment for commenting out blocks of text & for better verbatim
\usepackage{subfig} % make it possible to include more than one captioned figure/table in a single float
% These packages are all incorporated in the memoir class to one degree or another...
\usepackage[italian]{babel}

%%% HEADERS & FOOTERS
\usepackage{fancyhdr} % This should be set AFTER setting up the page geometry
\pagestyle{fancy} % options: empty , plain , fancy
\renewcommand{\headrulewidth}{0pt} % customise the layout...
\lhead{}\chead{}\rhead{}
\lfoot{}\cfoot{\thepage}\rfoot{}

%%% SECTION TITLE APPEARANCE
\usepackage{sectsty}
\allsectionsfont{\sffamily\mdseries\upshape} % (See the fntguide.pdf for font help)
% (This matches ConTeXt defaults)

%%% ToC (table of contents) APPEARANCE
\usepackage[nottoc,notlof,notlot]{tocbibind} % Put the bibliography in the ToC
\usepackage[titles,subfigure]{tocloft} % Alter the style of the Table of Contents
\renewcommand{\cftsecfont}{\rmfamily\mdseries\upshape}
\renewcommand{\cftsecpagefont}{\rmfamily\mdseries\upshape} % No bold!

\usepackage{amsmath}
\usepackage{amsfonts}
\usepackage{amsthm}

\theoremstyle{plain}% default
\newtheorem{thm}{Teorema}[section]
\newtheorem{lem}[thm]{Lemma}
\newtheorem{prop}[thm]{Proposizione}
\newtheorem*{cor}{Corollario}
\newtheorem*{KL}{Lemma di Klein}
\theoremstyle{definition}
\newtheorem{defn}{Definizione}[section]
\newtheorem{conj}{Congettura}[section]
\newtheorem{exmp}{Esempio}[section]
\theoremstyle{remark}
\newtheorem*{comm}{Commento}
\newtheorem*{note}{Nota}
\newtheorem{caso}{Caso}
\usepackage[mode=buildnew]{standalone}% requires -shell-escape
\usepackage{tikz}
\usepackage{pgfplots}
\pgfplotsset{width=10cm,compat=1.17}
\usepackage{pgfplotstable}

\usepackage{multirow}
% \usepackage[keeptemps]{pythontex}
\usepackage{float}
\usepackage{listings}
\lstdefinelanguage{JavaScript}{
  keywords={typeof, new, true, false, catch, function, return, null, catch, switch, var, if, in, while, do, else, case, break},
  keywordstyle=\color{blue}\bfseries,
  ndkeywords={class, export, boolean, throw, implements, import, this},
  ndkeywordstyle=\color{darkgray}\bfseries,
  identifierstyle=\color{black},
  sensitive=false,
  comment=[l]{//},
  morecomment=[s]{/*}{*/},
  commentstyle=\color{purple}\ttfamily,
  stringstyle=\color{red}\ttfamily,
  morestring=[b]',
  morestring=[b]"
}

\lstset{
   language=JavaScript,
   backgroundcolor=\color{lightgray},
   extendedchars=true,
   basicstyle=\footnotesize\ttfamily,
   showstringspaces=false,
   showspaces=false,
   numbers=left,
   numberstyle=\footnotesize,
   numbersep=9pt,
   tabsize=2,
   breaklines=true,
   showtabs=false,
   captionpos=b
}

\newfloat{lstfloat}{htbp}{lop}
\floatname{lstfloat}{Codice sorgente}
\def\lstfloatautorefname{Listato} % needed for hyperref/auroref


%%% The "real" document content comes below...

\title{Alcuni metodi iterativi per la ricerca di radici di funzioni}
\author{Gionata Massi}
\date{} % Activate to display a given date or no date (if empty),
         % otherwise the current date is printed 

\begin{document}
\maketitle

\thispagestyle{empty}%\frenchspacing

\tableofcontents

\section{Il problema}

Data una funzione $f : \mathbb{R} \to \mathbb{R}$, determinare un valore reale $\alpha$ tale che $f(\alpha) = 0$.

Usualmente consideriamo funzioni continue in $\mathbb{R}$ o almeno in un intevallo $[a, b] \subseteq \mathbb{R}$ chiuso e limitato in cui ricercare una radice.

\section{Esistenza delle radici}

Non tutte le funzioni ammettono radici, ad esempio $x \mapsto k$ e $x \mapsto (x + k)^2$, dove $k \neq 0$ (es: fig.~\ref{fig:no_zeri}).

\begin{figure}[ht]
    \centering
    \includestandalone{img/no_zeri}
    \caption{Funzioni che non intersecano l'asse $y = 0$}
    \label{fig:no_zeri}
\end{figure}
    
Altre funzioni hanno radici nei punti di massimo o di minimo locale (es: fig. \ref{fig:zero_estremante}).

\begin{figure}[ht]
    \centering
    \includestandalone{img/zero_estremanti}
    \caption{Funzioni che intersecano l'asse $y = 0$ in un estremante}
    \label{fig:zero_estremante}
\end{figure}

\subsection{Teorema degli zeri}

Per essere sicuri che una funzione ammetta almeno una radice richiediamo che la funzione assuma valori positivi e negativi in un certo intervallo e che sia continua.

\begin{thm}[Bolzano]
Se $f (x)$ è una funzione continua sull'intervallo limitato e chiuso $[a, b]$ e $f (a) \cdot f (b) < 0$, allora esiste almeno una radice di $f (x)$ nell'intervallo $[a, b]$.
\end{thm}

Se le ipotesi del teorema sono vere può esistere una sola radice oppure ce ne possono essere in numero finito o anche infinite (fig. \ref{fig:ipotesibolzano}).

\begin{figure}[ht]
    \centering
    \includestandalone{img/segni_discordi}
    \caption{Funzioni che assumono valori opposti agli estremi -1, 1}
    \label{fig:ipotesibolzano}
\end{figure}

Un metodo di ricerca delle radici, se convergente, restituirà una sola delle radici.
Si intuisce che maggiore è la pendenza della funzione in un intorno della radice, più è facile discriminare la radice. Se invece la pendenza è nulla o quasi, allora il problema si dice mal condizionato.

\section{Metodi numerici}

\subsection{Metodi diretti e metodi iterativi}

I \textbf{metodi diretti} sono algoritmi che, in assenza di errori di arrotondamento, forniscono la soluzione in un numero finito di operazioni.

I \textbf{metodi iterativi} sono algoritmi nei quali la soluzione è ottenuta come limite di una successione di soluzioni. Nella risposta fornita da un metodo iterativo è, quindi, presente usualmente un errore di troncamento.

\subsection{Ordine di convergenza}

Sia $x_k, k = 0, 1,\ldots$ una successione convergente al valore $\alpha$.

\begin{defn}[Errore all'iterazione $k$-esima]
L'errore per elemento della successione in posizione $k$ è il valore non negativo
$\varepsilon_k = |x_k - \alpha|$.
\end{defn}

\begin{defn}[Ordine di convergenza]
Se esistono un numero reale $p \geq 1$ e una costante reale positiva $C$
tale che
$\lim_{k\to+\infty} \frac{\varepsilon_{k+1}}{\varepsilon_{k}^p}=C$
allora la successione $x_k, k = 0, 1,\ldots$ ha ordine di convergenza $p$
e costante d'errore $C$.
\end{defn}

Si noti che nella definizione di ordine di convergenza, la costante $C$
è positiva, ossia strettamente maggiore di zero.

\subsection{Criteri di arresto}

I metodi iterativi sviluppano una successione di valori che deve essere troncata
per produrre un risultato.
Si posso usare vari criteri per arrestare un metodo numerico in base a vincoli
sull'approssimazione della soluzione cercata e sulle risorse di calcolo
(essenzialmente il tempo di calcolo) da impiegare.
I criteri di arresto comunemente adottati sono i seguenti:

\begin{description}
\item[Tolleranza sull'approssimazione della funzione]$|f(x_k)| < \tau_y$.
\item[Tolleranza assoluta sull'approssimazione di $\alpha$] $|x_{k + 1} - x_{k}| < \tau_x$.
\item[Tolleranza relativa sull'approssimazione di $\alpha$] $\frac{|x_{k + 1} - x_{k}|}{|x_{k+1}|} < \tau_r$.
\item[Numero di iterazioni] $k < k_M$. 
\end{description}

% Si veda Comincioli2004.

\section{Metodo di bisezione}

Sia $f (x)$ una funzione continua sull'intervallo limitato e chiuso $[a, b]$ con $f (a) \cdot f (b) < 0$.

L'algoritmo genera una successione di intervalli
$[a_k , b_k]$ con $f (a_k ) \cdot f (b_k ) < 0$ e con $[a_k , b_k ] \subset [a_{k-1}$ , $b_{k-1} ]$ e $|b_k - a_k | = \frac{1}{2}|b_{k-1} - a_{k-1} |$. 

Date due tolleranze $\epsilon_1$ , $\epsilon_2$ , l’algoritmo si arresta quando:

\begin{itemize}
    \item $|b_k - a_k | \leq \epsilon_1$, oppure
    \item $|f (\frac {a_k +b_k}{2} )| \leq \epsilon_2$, o infine
    \item $k > \text{nmax}$, ove nmax è un numero massimo di iterazioni fissato.
\end{itemize}

Per alleggerire la notazione usiamo $s_{a_k}$ per indicare $\mathrm{segno} \left(f\left(a_k\right)\right)$,
$s_{b_k}$ per $\mathrm{segno}(f(b_k))$ e
$s_k$ per $\mathrm{segno}(f (\frac {a_k +b_k}{2}))$, dove

\begin{equation}
    \mathrm{segno}(x) =
    \begin{cases}
        -1 & \text{se } x < 0, \\
         0 & \text{{se }} x = 0, \\
        +1 & \text{{se }} x > 0
    \end{cases}
\end{equation}

Per il calcolo di $\frac{a_k+b_k}{2}$ in virgola mobile si deve usare la formula: $a_k + (b_k - a_k) / 2$ in modo da ridurre gli errori di troncamento.

\subsection{Convergenza}

Il metodo è sempre convergente e si può calcolare l'ordine considerando l'errore
come l'ampiezza dell'intervallo di incertezza: $\varepsilon_k = \frac{|b - a|}{2^k}$.
Il rapporto tra due errori successivi per $p=1$ vale

$$\displaystyle \frac{\varepsilon_{k+1}}{\varepsilon_{k}^p}
=\frac{
    \frac{|b - a|}{2^{k+1}}
  }{
      \frac{|b - a|}{2^k}
  }
  = \frac{1}{2}$$

L'ordine di convergenza è uno con costante d'errore vale un mezzo.

\subsection{Codifica in JavaScript}

Si veda il listato \ref{lst:bisezione}.

\begin{lstfloat}
    \lstinputlisting{lst/bisezione.js}
    \caption{Descrizione in JavaScript del metodo di Bisezione}
    \label{lst:bisezione}
\end{lstfloat}

\subsection{Esempi}

Si vedano le tabb.~\ref{tbl:tab_bis_sqrt_6}, \ref{tbl:tab_bis_sin} e \ref{tbl:tab_bis_exp_mx} e le figg. \ref{fig:bis_sqrt_6} e \ref{fig:subplots_bis_sqrt_6}.

\begin{table}[!htbp]
    \begin{center}
\pgfplotstabletypeset[
	col sep=tab,
    every head row/.style={before row=\toprule,after row=\midrule},	% style the first row
	every last row/.style={after row=\bottomrule},	% style the last row
    every column/.style={dec sep align,precision=10}
]{tbl/tab_bis_sqrt_6.dat}
\end{center}        
\caption[]{Metodo dicotomico applicato a $x^2 -6$ nell'intervallo $[0, 6]$ con nmax = 10}
\label{tbl:tab_bis_sqrt_6}
\end{table}

\begin{figure}[!htbp]
    \centering
    \includestandalone{img/iter_bis_sqrt_6}
    \caption{Successione delle soluzioni del metodo dicotomico applicato a $x^2 -6$ nell'intervallo $[0, 6]$}
    \label{fig:bis_sqrt_6}
\end{figure}

\begin{figure}[!htbp]
    \centering
    \includestandalone{img/iter_bis_sqrt_6_stacked}
    \caption{Passi del metodo dicotomico applicato a $x^2 -6$ nell'intervallo $[0, 6]$}
    \label{fig:subplots_bis_sqrt_6}
\end{figure}


\begin{table}[!htbp]
    \begin{center}
        \pgfplotstabletypeset[
            col sep=tab,
            every head row/.style={before row=\toprule,after row=\midrule},	% style the first row
            every last row/.style={after row=\bottomrule},	% style the last row
            every column/.style={dec sep align,precision=10}
        ]{tbl/tab_bis_sin.dat}
    \end{center}        
    \caption{Metodo dicotomico applicato a $sin(x)$ nell'intervallo $[3, 3.2]$ con nmax = 10}
    \label{tbl:tab_bis_sin}
\end{table}

\begin{table}[!htbp]
    \begin{center}
        \pgfplotstabletypeset[
            col sep=tab,
            every head row/.style={before row=\toprule,after row=\midrule},	% style the first row
            every last row/.style={after row=\bottomrule},	% style the last row
            every column/.style={dec sep align,precision=10}
            %columns/.style={sci,sci subscript,sci zerofill,dec sep align}
            %every first column/.style={column type/.add={|}{}},	% style the first column
            %every last column/.style={column type/.add={}{|}},	% style the last column
            %columns/C/.style = {column type/.add={|}{|}}	% style the designated column
        ]{tbl/tab_bis_exp_mx.dat}
    \end{center}        
    \caption{Metodo dicotomico applicato a $e^{e^{-x}}-x$ nell'intervallo $[0, 1]$ con nmax = 10}
    \label{tbl:tab_bis_exp_mx}
\end{table}


\section{Iterazioni di punto fisso}

In generale si può costruire un metodo iterativo cercando un punto fisso di una funzione $\Phi(x)$,
costruita in moda che si annulli ne punto desiderato, un valore $\bar{x}$ tale che $\Phi(\bar{x}) = \bar{x}$.

Il punto fisso è calcolato tramite l'applicazione ripetuta della regola di ricorrenza:

$$x_{k+1} = \Phi(x_k)$$

\subsection{Approssimazioni con rette}

Una retta è definita da una funzione del tipo $r(x) = m x + q$.
Se imponiamo il passaggio per il punto $(x_k, f(x_k))$ della funzione di cui cerchiamo una radice, il fascio di rette sarà:

$$r(x) - f(x_k) = m (x - x_k)$$

Possiamo generare delle iterazioni andando a fissare il coefficiente angolare ad ogni iterazione e determinando l'intersezione della retta con l'asse delle ascisse.

$$r(x_{k+1}) - f(x_k) = m_k (x_{k+1} - x_k)$$

Risolvendo per $$r(x_{k+1}) = 0$$ si ha

\begin{equation}
    x_{k+1} = x_k - \frac{f(x_k)}{m_k}
\end{equation}

\begin{figure}
    \begin{center}
        \includestandalone{img/approx_rette}
        \caption{Approssimazione di una funzione con una retta passante per ($x_k$, $f(x_k)$)}
        \label{fig:retta_approx}
    \end{center}
\end{figure}

\subsection{Metodo delle corde}

Si considerino due punti $a=x_0$ e $b=x_1$ tali da soddisfare le ipotesi del teorema di Bolzano.
È possibile costruire una successione che per ogni $k \geq 0$  il punto $x_{k+1}$ sia lo zero della retta passante per il punto
($x_{k}$, $f(x_{k})$) e di coefficiente angolare

$$\displaystyle m_k = \frac{f(a)  - f(x_k)}{a - x_k}.$$

L'iterata ha equazione:

$$x_{k+1} = x_k - f(x_k) \frac{a  - x_k)}{f(a) - f(x_k)}.$$

\subsubsection{Convergenza}

Il metodo non necessariamente converge.
Può oscillare (vedi tab.~\ref{tbl:tab_cor_osc}) o divergere.


\subsubsection{Codifica in JavaScript}

Si veda il listato \ref{lst:corde}.

\begin{lstfloat}
    \lstinputlisting{lst/corde.js}
    \caption{Descrizione in JavaScript del metodo delle corde}
    \label{lst:corde}
\end{lstfloat}

\subsubsection{Esempi}

Si vedano le tabb.~\ref{tbl:tab_cor_sqrt_6}, \ref{tbl:tab_cor_sin} e \ref{tbl:tab_cor_exp_mx} e la fig. \ref{fig:cor_sqrt_6}.

\begin{table}
    \begin{center}
\pgfplotstabletypeset[
	col sep=tab,
    every head row/.style={before row=\toprule,after row=\midrule},	% style the first row
	every last row/.style={after row=\bottomrule},	% style the last row
    every column/.style={dec sep align,precision=10}
]{tbl/tab_bis_sqrt_6.dat}
\end{center}        
\caption[]{Metodo delle corde applicato a $x^2 -6$ nell'intervallo $[0, 6]$ con nmax = 10}
\label{tbl:tab_cor_sqrt_6}
\end{table}

\begin{figure}[ht]
    \centering
    \includestandalone{img/iter_cor_sqrt_6}
    \caption{Successione delle soluzioni del metodo delle corde applicato a $x^2 -6$ nell'intervallo $[0, 6]$}
    \label{fig:cor_sqrt_6}
\end{figure}

\begin{table}
    \begin{center}
        \pgfplotstabletypeset[
            col sep=tab,
            every head row/.style={before row=\toprule,after row=\midrule},	% style the first row
            every last row/.style={after row=\bottomrule},	% style the last row
            every column/.style={dec sep align,precision=10}
        ]{tbl/tab_cor_sin.dat}
    \end{center}        
    \caption[]{Metodo delle corde applicato a $sin(x)$ nell'intervallo $[3, 3.2]$ con nmax = 10}
    \label{tbl:tab_cor_sin}
\end{table}

\begin{table}
    \begin{center}
        \pgfplotstabletypeset[
            col sep=tab,
            every head row/.style={before row=\toprule,after row=\midrule},	% style the first row
            every last row/.style={after row=\bottomrule},	% style the last row
            every column/.style={dec sep align,precision=10}
            %columns/.style={sci,sci subscript,sci zerofill,dec sep align}
            %every first column/.style={column type/.add={|}{}},	% style the first column
            %every last column/.style={column type/.add={}{|}},	% style the last column
            %columns/C/.style = {column type/.add={|}{|}}	% style the designated column
        ]{tbl/tab_cor_exp_mx.dat}
    \end{center}        
    \caption[]{Metodo delle corde applicato a $e^{e^{-x}}-x$ nell'intervallo $[0, 1]$ con nmax = 10}
    \label{tbl:tab_cor_exp_mx}
\end{table}

\begin{table}
    \begin{center}
\pgfplotstabletypeset[
	col sep=tab,
    every head row/.style={before row=\toprule,after row=\midrule},	% style the first row
	every last row/.style={after row=\bottomrule},	% style the last row
    every column/.style={dec sep align,precision=10}
]{tbl/tab_cor_osc.dat}
\end{center}        
\caption[]{Metodo delle corde applicato a $x^2 -1$ nell'intervallo $[0, 2]$ con nmax = 10}
\label{tbl:tab_cor_osc}
\end{table}


\subsection{Metodo delle secanti}

Dati due punti iniziali $x_0$ e $x_1$, si considera la secante passante per i due punti dati.

In generale il coefficiente angolare $m_k$ è calcolato come:

$$m_k = \frac{f(x_k) - f(x_{k-1})}{x_k - x_{k-1}}.$$

L'iterata è

$$x_{k+1} = x_k - f(x_k) \cdot \frac{x_k - x_{k-1}}{f(x_k) - f(x_{k-1})}.$$

\subsubsection{Convergenza}

Il metodo non necessariamente converge.
Può oscillare o divergere.

\subsubsection{Codifica in JavaScript}

\begin{lstfloat}
    \lstinputlisting{lst/secanti.js}
    \caption{Descrizione in JavaScript del metodo delle secanti}
    \label{lst:secanti}
\end{lstfloat}

\subsubsection{Esempi}

Si vedano gli esempi nelle tabb.~\ref{tbl:tab_sec_sqrt_6}, \ref{tbl:tab_sec_sin} e \ref{tbl:tab_sec_exp_mx} e la fig.~\ref{fig:sec_sqrt_6}.


\begin{table}
    \begin{center}
\pgfplotstabletypeset[
	col sep=tab,
    every head row/.style={before row=\toprule,after row=\midrule},	% style the first row
	every last row/.style={after row=\bottomrule},	% style the last row
    every column/.style={dec sep align,precision=10}
]{tbl/tab_bis_sqrt_6.dat}
\end{center}        
\caption[]{Metodo delle secanti applicato a $x^2 -6$ nell'intervallo $[0, 6]$ con nmax = 10}
\label{tbl:tab_sec_sqrt_6}
\end{table}

\begin{figure}[ht]
    \centering
    \includestandalone{img/iter_cor_sqrt_6}
    \caption{Successione delle soluzioni del metodo delle secanti applicato a $x^2 -6$ nell'intervallo $[0, 6]$}
    \label{fig:sec_sqrt_6}
\end{figure}

\begin{table}
    \begin{center}
        \pgfplotstabletypeset[
            col sep=tab,
            every head row/.style={before row=\toprule,after row=\midrule},	% style the first row
            every last row/.style={after row=\bottomrule},	% style the last row
            every column/.style={dec sep align,precision=10}
        ]{tbl/tab_cor_sin.dat}
    \end{center}        
    \caption[]{Metodo delle secanti applicato a $sin(x)$ nell'intervallo $[3, 3.2]$ con nmax = 10}
    \label{tbl:tab_sec_sin}
\end{table}

\begin{table}
    \begin{center}
        \pgfplotstabletypeset[
            col sep=tab,
            every head row/.style={before row=\toprule,after row=\midrule},	% style the first row
            every last row/.style={after row=\bottomrule},	% style the last row
            every column/.style={dec sep align,precision=10}
            %columns/.style={sci,sci subscript,sci zerofill,dec sep align}
            %every first column/.style={column type/.add={|}{}},	% style the first column
            %every last column/.style={column type/.add={}{|}},	% style the last column
            %columns/C/.style = {column type/.add={|}{|}}	% style the designated column
        ]{tbl/tab_cor_exp_mx.dat}
    \end{center}        
    \caption[]{Metodo delle secanti applicato a $e^{e^{-x}}-x$ nell'intervallo $[0, 1]$ con nmax = 10}
    \label{tbl:tab_sec_exp_mx}
\end{table}


\subsection{Metodo delle tangenti}

Si approssima la funzione $f(x)$ con la retta $r(x) - f(x_k) = f'(x_k) (x - x_k)$ tangente ad essa in $(x_k, f(x_k))$.

L'iterata assume la forma:

$$x_{k+1} = x_k - \frac{f(x_k)}{f'{x_k}}.$$

Si veda la fig.~\ref{fig:retta_tangente}

\begin{figure}[!htbp]
    \begin{center}
        \includestandalone{img/tan_generale}
        \caption{Approssimazione di una funzione con la retta tangente passante in ($x_k$, $f(x_k)$)}
        \label{fig:retta_tangente}
    \end{center}
\end{figure}


\subsubsection{Convergenza}

Il metodo non necessariamente converge.
Può oscillare o divergere.

\subsubsection{Codifica in JavaScript}


Si vedanil listato \ref{lst:tangenti}.

\begin{lstfloat}
    \lstinputlisting{lst/tangenti.js}
    \caption{Descrizione in JavaScript del metodo delle tangenti}
    \label{lst:tangenti}
\end{lstfloat}

\subsubsection{Esempi}

Si vedano le tabb.~\ref{tbl:tab_tan_sqrt_6}, \ref{tbl:tab_tan_sin} e \ref{tbl:tab_tan_exp_mx} e le fig. \ref{fig:tan_sqrt_6} e \ref{fig:geo_tan_sqrt_6}.

Un esempio che riassume il metodo è in fig.~\ref{fig:geo_tab_completo}.

\begin{table}[!htbp]
    \begin{center}
\pgfplotstabletypeset[
	col sep=tab,
    every head row/.style={before row=\toprule,after row=\midrule},	% style the first row
	every last row/.style={after row=\bottomrule},	% style the last row
    columns/0/.style ={dec sep align,column name=$k$, precision=1},
    columns/1/.style ={dec sep align,column name=$x_k$, precision=6},
    columns/2/.style ={dec sep align,column name=$f_k$,	precision=4},
    columns/3/.style ={dec sep align,column name=$m_k$, precision=4},
    columns/4/.style ={dec sep align,column name=$|x_{k+1}-x_{k}|$, precision=4},
]{tbl/tab_tan_sqrt_6.dat}
\end{center}        
\caption[]{Metodo delle tangenti applicato a $x^2 -6$ con stima iniziale 3}
\label{tbl:tab_tan_sqrt_6}
\end{table}

\begin{figure}[!htbp]
    \centering
    \includestandalone{img/iter_tan_sqrt_6}
    \caption{Successione delle soluzioni del metodo delle tangenti applicato a $x^2 -6$}
    \label{fig:tan_sqrt_6}
\end{figure}

\begin{figure}[!htbp]
    \centering
    \includestandalone{img/geo_tan_sqrt_6}
    \caption{Successione delle soluzioni del metodo delle tangenti applicato a $x^2 -6$}
    \label{fig:geo_tan_sqrt_6}
\end{figure}


\begin{table}[!htbp]
    \begin{center}
        \pgfplotstabletypeset[
            col sep=tab,
            every head row/.style={before row=\toprule,after row=\midrule},	% style the first row
            every last row/.style={after row=\bottomrule},	% style the last row
            every column/.style={dec sep align,precision=10}
            %every first column/.style={column type/.add={|}{}},	% style the first column
            %every last column/.style={column type/.add={}{|}},	% style the last column
            %columns/C/.style = {column type/.add={|}{|}}	% style the designated column
        ]{tbl/tab_tan_sin.dat}
    \end{center}        
    \caption[]{Metodo delle tangenti applicato a $sin(x)$ con stima iniziale 3,1}
    \label{tbl:tab_tan_sin}
\end{table}

\begin{table}[!htbp]
    \begin{center}
        \pgfplotstabletypeset[
            col sep=tab,
            every head row/.style={before row=\toprule,after row=\midrule},	% style the first row
            every last row/.style={after row=\bottomrule},	% style the last row
            every column/.style={dec sep align,precision=10}
            %columns/.style={sci,sci subscript,sci zerofill,dec sep align}
            %every first column/.style={column type/.add={|}{}},	% style the first column
            %every last column/.style={column type/.add={}{|}},	% style the last column
            %columns/C/.style = {column type/.add={|}{|}}	% style the designated column
        ]{tbl/tab_tan_exp_mx.dat}
    \end{center}        
    \caption[]{Metodo delle tangenti applicato a $e^{e^{-x}}-x$ con stima iniziale 0,5}
    \label{tbl:tab_tan_exp_mx}
\end{table}

\begin{figure}[!htbp]
    \centering
    \includestandalone{img/geo_tan}
    \caption{Riepilogo sul metodo delle tangenti applicato a $e^{0.9x - x^2}$}
    \label{fig:geo_tab_completo}
\end{figure}


\subsubsection{Esempio: algoritmo del reciproco}

Si vuole cercare il reciproco del valore $\nu$ come $\alpha = \frac{1}{\nu}$ con il metodo delle tangenti.

Per prima cosa occorre trasformare il problema con una funzione che si annulla in $\frac{1}{\nu}$.

Scegliamo $$f(x) = \nu - \frac{1}{x}$$ che applicata al reciproco di $\nu$ produce
$f(\frac{1}{\nu}) = \nu - \frac{1}{\frac{1}{\nu}} = \nu - \nu = 0$. 

La derivata prima assume la forma $f'(x) = \frac{1}{x^2}$ e

$$\frac{f(x)}{f'(x)} = \frac{\nu - \frac{1}{x}}{\frac{1}{x^2}} = \nu x^2 - x.$$

L'iterata del metodo delle tangenti è:

$$x_{k+1} = x_k - (\nu x_k^2 - x_k) = 2 x_k - \nu x_k^2  = x_k \cdot (2 - \nu \cdot x_k).$$

Si noti che per calcolare il reciproco di un numero sono sufficienti le operazioni di moltiplicazione e sottrazione.
Per la moltiplicazione occorrono le operazioni primitive di scorrimento e addizione e per la sottrazione quelle di negazione bit a bit e di addizione (basterebbe il solo incremento unitario).

Nota: queste proprietà permettono di realizzare le CPU senza l'operazione di divisione.


\subsubsection{Esempio: metodo di Erone per la radice quadrata}

Si consideri il problema di determinare la radice quadrata di un numero reale $r$.

Se dal punto di vista algebrico possiamo vedere il problema come determinare uno zero della funzione $f(x) = x^2 - r$,
da un punto di vista geometrico possiamo porci il problema di trovare il lato di un quadrato di area $r$.


L'idea di Erone di Alessandria è quella fornire una stima della radice quadrata, che chiamiamo $x_k$ e che
rappresentiamo geometricamente come la base di un rettangolo. Chiamiamo $h_k = \frac{r}{x_k}$ l'altezza del rettangolo
di area $r$ e base $x_k$.
Se $x_k$ e $h_k$ fossero molto vicini tra di loro, allora abbiamo trovato una buona stima per $\sqrt{r}$, altrimenti
occorre migliorare la stima. Per fare ciò consideriamo il valor medio tra $x_k$ e $h_k$.

L'iterata del metodo, ossia la stima migliore, ha equazione:

$$x_{k+1} = \frac{x_k+\frac{r}{x_k}}{2}$$

che può essere riscritta come:

$$x_{k+1} = \frac{x_k+ \frac{r}{x_k}}{2} = \frac{x_k^2 + r}{2x_k}$$

Vogliamo dimostrare che la traccia di esecuzione del metodo di Erone di Alessandria è la stessa del metodo
di Newton applicato alla funzione $f(x) = x^2 - r$ calcolando l'iterata:

$$x_{k+1} = x_k - \frac{f(x_k)}{f'(x_k)} =
            x_k - \frac{x_k^2 - r}{2 x_k} =
            \frac{2 x_k^2 - ( x_k^2 - r)}{2 x_k} =
            \frac{x_k^2 + r}{2 x_k}$$

Le iterate assumono lo stesso valore, come volevasi dimostrare!

Si vedano 

\begin{table}[!htbp]
    \begin{center}
\pgfplotstabletypeset[
	col sep=tab,
    every head row/.style={before row=\toprule,after row=\midrule},	% style the first row
	every last row/.style={after row=\bottomrule},	% style the last row
    every column/.style={dec sep align,precision=10}
	%every first column/.style={column type/.add={|}{}},	% style the first column
	%every last column/.style={column type/.add={}{|}},	% style the last column
	%columns/C/.style = {column type/.add={|}{|}}	% style the designated column
]{tbl/tab_erone_sqrt_6.dat}
\end{center}        
\caption[]{Metodo di Erone di Alessandria per il calcolo di $\sqrt{6}$ con stima iniziale 3 e nmax = 10}
\label{tab:erone_sqrt_6}
\end{table}

\begin{figure}[!htbp]
    \centering
    \includestandalone{img/geo_erone_1}
    \caption{Successione delle soluzioni del metodo di Erone per il calcolo di $\sqrt{6}$ con stima iniziale 3}
    \label{fig:geo_erone_sqrt_6}
\end{figure}

\section{Estensioni per il calcolo degli estremanti relativi}

L'analisi matematica fornisce delle condizioni che mettono in relazione i punti
di massimo e minimo relativo con quelli \textit{stazionari}.

\begin{thm}[Teorema di Fermat sui punti stazionari]
Sia $f :(a, b) \to \mathbb{R}$ una funzione e si supponga che $x_0 \in  (a, b)$
sia un punto di estremo locale di $f$. Se $f$ è derivabile nel punto $x_0$,
allora $f^{\prime}(x_0) = 0$.
\end{thm}

Trovare un estremante di una funzione, se è derivabile con condizioni che
dipendono dal metodo scelto e la derivata è nota in forma simbolica o numerica,
si riconduce al problema di determinare una radice della derivata prima.

Tra i metodi più interessanti si trova quello di Newton.

\subsection{Metodo di Newton per l'ottimizzazione}

Possiamo applicare il metodo di Newton per la ricerca delle radici ad una
funzione $g(x) = f'(x)$ per trovare un punto stazionario. Si richiede che $f$
sia continua e derivabile almeno due volte in un intorno di $x_0$, stima iniziale
del punto estremante.

$$x_{k + 1} = x_k - \frac{g(x_k)}{g'(x_k)} = x_k - \frac{f'(x_k)}{f''(x_k)}$$

\subsubsection[Interpretazione geometrica]{Interpretazione geometrica del metodo di Newton per l'ottimizzazione}

Possiamo sviluppare il metodo in modo geometrico, approssimando $f(x_k)$ con una parabola $p(x) = a x^2 + bx + c$
passante per il punto $\left(x_k, f(x_k)\right)$, con la stessa tangente nel punto $x_k$ e stessa derivata seconda.
Trovati i coefficienti $a, b$ e $c$ della parabola, possiamo approssimare il vertice della funzione $f$
con il vertice della parabola, che sappiamo avere proiezione sull'asse delle ascisse $x_{k+1} = \frac{-b}{2a}$.

Per calcolare $a$ e $b$, gli unici due parametri usati per determinare il vertice, imponiamo le condizioni sulle
derivate prime e seconda.

\systeme{f'(x_k) = 2a x_k + b,
         f''(x_k) = 2a}

Dal sistema abbiamo esplicitato $2a = f''(x_k)$ e dobbiamo ricavare $b$ dalla prima equazione.

$b = f'(x_k) - 2a x_k = f'(x_k) - f''(x_k) x_k$.

L'iterata del metodo geometrico che a partire dal punto $x_k$ approssima l'estremante di $f$
con il vertice di $p$ ha equazione:

$$x_{k+1} = \frac{-b}{2a} = \frac{f''(x_k) x_k - f'(x_k)}{f''(x_k)} = x_k - \frac{f'(x_k)}{f''(x_k)}$$

Per completezza, possiamo determinare tutti i coefficienti di $p$:

\systeme{f(x_k) = a x_k^2 + b x_k + c,
         f'(x_k) = 2a x_k + b,
         f''(x_k) = 2a}

Sostituendo nella prima equazione $a$ e $b$ si ha:

$$c = - \frac{1}{2} f''(x_k) - f'(x_k) x_k + f''(x_k) x_k^2 + f(x_k) = f(x_k)  - f'(x_k) x_k + \frac{1}{2} f''(x_k) x_k^2$$


Questa è l'interpretazione geometrica del metodo di Newton applicato alla derivata prima.

\subsection{Esempio}

Consideriamo la funzione $f(x) = \frac{1}{3} x^3 + 2 x^2 + x + 6$, di cui è agevole
calcolare le derivate prima, $f'(x) = x^2 + 4 x + 1$, e seconda, $f''(x) = 2x + 4$.

Analiticamente è facile trovare gli estremanti in $x_M = -2 \pm \sqrt{3}$.

Numericamente potremmo valutare la funzione in più punti e magari potremmo osservare
che gli estremanti si trovano nell'intervallo $[-5, 2]$ e consideriamo una volta il
punto -3 e l'altra -1 come stime iniziali.

Possiamo riscrivere l'iterata come:

$$x_{k+1} = x_k - \frac{x^2 + 4 x + 1}{2x + 4}$$

I parametri della parabola sono:

\systeme{f(x_k) = a x_k^2 + b x_k + c,
         f'(x_k) = 2a x_k + b,
         f''(x_k) = 2a}

\systeme{c = \frac{1}{3} x_k^3 + 6,
b = 1 - x_k^2 ,
a = x_k + 2}

La risoluzione con due punti iniziali diversi sono nelle tabb.~\ref{tbl:par_es1} e \ref{tbl:par_es2}.
Per confrontare la soluzione geometrica (sostituzione del massimo con il vertice della parabola),
funzione derivata e sua tangente, si veda la fig.~\ref{fig:geo_tan_impilato}.

\begin{table}[!htbp]
    \begin{center}
\pgfplotstabletypeset[
	col sep=tab,
    every head row/.style={before row=\toprule,after row=\midrule},	% style the first row
	every last row/.style={after row=\bottomrule},	% style the last row
    every column/.style={dec sep align,precision=10}
]{tbl/tab_par_es1.dat}
\end{center}        
\caption[]{Metodo delle parabole con nmax = 5. $x_0 = -3$}
\label{tbl:par_es1}
\end{table}

\begin{table}[!htbp]
    \begin{center}
\pgfplotstabletypeset[
	col sep=tab,
    every head row/.style={before row=\toprule,after row=\midrule},	% style the first row
	every last row/.style={after row=\bottomrule},	% style the last row
    every column/.style={dec sep align,precision=10}
]{tbl/tab_par_es2.dat}
\end{center}        
\caption[]{Metodo delle parabole con nmax = 5. $x_0 = -1$}
\label{tbl:par_es2}
\end{table}

\begin{figure}[!htbp]
    \centering
    \includestandalone{img/geo_par_tan_es_1}
    \caption{Successione delle soluzioni del metodo delle parabole}
    \label{fig:geo_tan_impilato}
\end{figure}


\subsection{Altri metodi}

\subsubsection{Metodo di Steffensen}

Le derivate sono complesse da calcolare quando non sono
note in forma esplicita ma è possibile ottenere un metodo
di ordine di convergenza due applicando più valutazini della funzione
senza ricorrere alla derivazione.

$$x_{k+1} = x_k -\frac{f(x_k)}{g(x_k)};\quad g(x_k) =\frac{f(x_k + f(x_k )) - f(x_k )}{f(x_k)}$$

$$x_{k+1} = x_k - \frac{f(x_k)^2}{f(x_k + f(x_k )) - f(x_k )}$$

Si noti che la valutazione della funzione $f$ in $x_k$ è molto vicina a zero.
Per apprezzare la funzione $g$ approssimi $f'$ riscriviamo $g(x_k)$ con
$h = f(x_k)$:

$$g(x_k) =\frac{f(x_k + h(x_k )) - f(x_k )}{f(x_k)}$$.

\subsubsection{Metodo di Muller}

Un altro modo di approssimare la funzione $f$ è quella di usare una parabola.
Nel metodo di Muller si stimano tre punti iniziali e si determina la parabola
passante per essi. Dalla soluzione dell'equazione di secondo grado si ottengo
due punti che si usano come nuove stime.
Risolvere le equazioni secondo grado, però, può portare a soluzioni inesistenti.

\section{Estensioni per il calcolo degli estremanti relativi}

L'analisi matematica fornisce delle condizioni che mettono in relazione i punti
di massimo e minimo relativo con quelli \textit{stazionari}.

\begin{thm}[Teorema di Fermat sui punti stazionari]
Sia $f :(a, b) \to \mathbb{R}$ una funzione e si supponga che $x_0 \in  (a, b)$
sia un punto di estremo locale di $f$. Se $f$ è derivabile nel punto $x_0$,
allora $f^{\prime}(x_0) = 0$.
\end{thm}

Trovare un estremante di una funzione, se è derivabile con condizioni che
dipendono dal metodo scelto e la derivata è nota in forma simbolica o numerica,
si riconduce al problema di determinare una radice della derivata prima.

Tra i metodi più interessanti si trova quello di Newton.

\subsection{Metodo di Newton per l'ottimizzazione}

Possiamo applicare il metodo di Newton per la ricerca delle radici ad una
funzione $g(x) = f'(x)$ per trovare un punto stazionario. Si richiede che $f$
sia continua e derivabile almeno due volte in un intorno di $x_0$, stima iniziale
del punto estremante.

$$x_{k + 1} = x_k - frac{g(x_k)}{g'(x_k)} = x_k - \frac{f'(x_k)}{f''(x_k)}$$

\begin{subsection}{Esempio}

Consideriamo la funzione $f(x) = \frac{1}{3} x^3 + 2 x^2 + x + 6$, di cui è agevole
calcolare le derivate prima, $f'(x) = x^2 + 4 x + 1$, e seconda, $f''(x) = 2x + 4$.

Analiticamente è facile trovare gli estremanti in $x_M = -2 \pm \sqrt{3}$.

Numericamente potremmo valutare la funzione in più punti e magari potremmo osservare
che gli estremanti si trovano nell'intervallo $[-5, 2]$ e consideriamo una volta il
punto -2 e l'altra -1 come stime iniziali.




\end{document}