\subsection{Metodo delle tangenti}

Si approssima la funzione $f(x)$ con la retta $r(x) - f(x_k) = f'(x_k) (x - x_k)$ tangente ad essa in $(x_k, f(x_k))$.

L'iterata assume la forma:

$$x_{k+1} = x_k - \frac{f(x_k)}{f'{x_k}}.$$

\subsubsection{Convergenza}

Il metodo non necessariamente converge.
Può oscillare o divergere.

\subsubsection{Codifica in JavaScript}


Si vedanil listato \ref{lst:tangenti}.

\begin{lstfloat}
    \lstinputlisting{lst/tangenti.js}
    \caption{Descrizione in JavaScript del metodo delle tangenti}
    \label{lst:tangenti}
\end{lstfloat}

\subsubsection{Esempi}

Si vedano le tabb.~\ref{tbl:tab_tan_sqrt_6}, \ref{tbl:tab_tan_sin} e \ref{tbl:tab_tan_exp_mx} e la fig. \ref{fig:tan_sqrt_6}.

\begin{table}
    \begin{center}
\pgfplotstabletypeset[
	col sep=tab,
    every head row/.style={before row=\toprule,after row=\midrule},	% style the first row
	every last row/.style={after row=\bottomrule},	% style the last row
    every column/.style={dec sep align,precision=10}
	%every first column/.style={column type/.add={|}{}},	% style the first column
	%every last column/.style={column type/.add={}{|}},	% style the last column
	%columns/C/.style = {column type/.add={|}{|}}	% style the designated column
]{tbl/tab_tan_sqrt_6.dat}
\end{center}        
\caption[]{Metodo delle tangenti applicato a $x^2 -6$ con stima iniziale 3 e nmax = 10}
\end{table}

\begin{figure}[ht]
    \centering
    \includestandalone{img/iter_tan_sqrt_6}
    \caption{Successione delle soluzioni del metodo delle tangenti applicato a $x^2 -6$ nell'intervallo $[0, 6]$}
    \label{fig:tan_sqrt_6}
\end{figure}

\begin{table}
    \begin{center}
        \pgfplotstabletypeset[
            col sep=tab,
            every head row/.style={before row=\toprule,after row=\midrule},	% style the first row
            every last row/.style={after row=\bottomrule},	% style the last row
            every column/.style={dec sep align,precision=10}
            %every first column/.style={column type/.add={|}{}},	% style the first column
            %every last column/.style={column type/.add={}{|}},	% style the last column
            %columns/C/.style = {column type/.add={|}{|}}	% style the designated column
        ]{tbl/tab_tan_sin.dat}
    \end{center}        
    \caption[]{Metodo delle tangenti applicato a $sin(x)$ con stima iniziale 3,1 e nmax = 10}
\end{table}

\begin{table}
    \begin{center}
        \pgfplotstabletypeset[
            col sep=tab,
            every head row/.style={before row=\toprule,after row=\midrule},	% style the first row
            every last row/.style={after row=\bottomrule},	% style the last row
            every column/.style={dec sep align,precision=10}
            %columns/.style={sci,sci subscript,sci zerofill,dec sep align}
            %every first column/.style={column type/.add={|}{}},	% style the first column
            %every last column/.style={column type/.add={}{|}},	% style the last column
            %columns/C/.style = {column type/.add={|}{|}}	% style the designated column
        ]{tbl/tab_tan_exp_mx.dat}
    \end{center}        
    \caption[]{Metodo delle tangenti applicato a $e^{e^{-x}}-x$ con stima iniziale 0,5 e nmax = 10}
\end{table}


\subsubsection{Esempio: algoritmo del reciproco}

Si vuole cercare il reciproco del valore $\nu$ come $\alpha = \frac{1}{\nu}$ con il metodo delle tangenti.

Per prima cosa occorre trasformare il problema con una funzione che si annulla in $\frac{1}{\nu}$.

Scegliamo $$f(x) = \nu - \frac{1}{x}$$ che applicata al reciproco di $\nu$ produce
$f(\frac{1}{\nu}) = \nu - \frac{1}{\frac{1}{\nu}} = \nu - \nu = 0$. 

La derivata prima assume la forma $f'(x) = \frac{1}{x^2}$ e

$$\frac{f(x)}{f'(x)} = \frac{\nu - \frac{1}{x}}{\frac{1}{x^2}} = \nu x^2 - x.$$

L'iterata del metodo delle tangenti è:

$$x_{k+1} = x_k - (\nu x_k^2 - x_k) = 2 x_k - \nu x_k^2  = x_k \cdot (2 - \nu \cdot x_k).$$

Si noti che per calcolare il reciproco di un numero sono sufficienti le operazioni di moltiplicazione e sottrazione.
Per la moltiplicazione occorrono le operazioni primitive di scorrimento e addizione e per la sottrazione quelle di negazione bit a bit e di addizione (basterebbe il solo incremento unitario).

Nota: queste proprietà permettono di realizzare le CPU senza l'operazione di divisione.

