\subsection{Metodo delle tangenti}

Si approssima la funzione $f(x)$ con la retta $r(x) - f(x_k) = f'(x_k) (x - x_k)$ tangente ad essa in $(x_k, f(x_k))$.

L'iterata assume la forma:

$$x_{k+1} = x_k - \frac{f(x_k)}{f'{x_k}}.$$

Si veda la fig.~\ref{fig:retta_tangente}

\begin{figure}[!htbp]
    \begin{center}
        \includestandalone{img/tan_generale}
        \caption{Approssimazione di una funzione con la retta tangente passante in ($x_k$, $f(x_k)$)}
        \label{fig:retta_tangente}
    \end{center}
\end{figure}


\subsubsection{Convergenza}

Il metodo non necessariamente converge.
Può oscillare o divergere.

\subsubsection{Codifica in JavaScript}


Si vedanil listato \ref{lst:tangenti}.

\begin{lstfloat}
    \lstinputlisting{lst/tangenti.js}
    \caption{Descrizione in JavaScript del metodo delle tangenti}
    \label{lst:tangenti}
\end{lstfloat}

\subsubsection{Esempi}

Si vedano le tabb.~\ref{tbl:tab_tan_sqrt_6}, \ref{tbl:tab_tan_sin} e \ref{tbl:tab_tan_exp_mx} e le fig. \ref{fig:tan_sqrt_6} e \ref{fig:geo_tan_sqrt_6}.

Una esempio che riassume il metodo è in fig.~\ref{fig:geo_tab}.

\begin{table}[!htbp]
    \begin{center}
\pgfplotstabletypeset[
	col sep=tab,
    every head row/.style={before row=\toprule,after row=\midrule},	% style the first row
	every last row/.style={after row=\bottomrule},	% style the last row
    columns/0/.style ={dec sep align,column name=$k$, precision=1},
    columns/1/.style ={dec sep align,column name=$x_k$, precision=6},
    columns/2/.style ={dec sep align,column name=$f_k$,	precision=4},
    columns/3/.style ={dec sep align,column name=$m_k$, precision=4},
    columns/4/.style ={dec sep align,column name=$|x_{k+1}-x_{k}|$, precision=4},
]{tbl/tab_tan_sqrt_6.dat}
\end{center}        
\caption[]{Metodo delle tangenti applicato a $x^2 -6$ con stima iniziale 3 e nmax = 10}
\end{table}

\begin{figure}[!htbp]
    \centering
    \includestandalone{img/iter_tan_sqrt_6}
    \caption{Successione delle soluzioni del metodo delle tangenti applicato a $x^2 -6$}
    \label{fig:tan_sqrt_6}
\end{figure}

\begin{figure}[!htbp]
    \centering
    \includestandalone{img/geo_tan_sqrt_6}
    \caption{Successione delle soluzioni del metodo delle tangenti applicato a $x^2 -6$}
    \label{fig:geo_tan_sqrt_6}
\end{figure}


\begin{table}[!htbp]
    \begin{center}
        \pgfplotstabletypeset[
            col sep=tab,
            every head row/.style={before row=\toprule,after row=\midrule},	% style the first row
            every last row/.style={after row=\bottomrule},	% style the last row
            every column/.style={dec sep align,precision=10}
            %every first column/.style={column type/.add={|}{}},	% style the first column
            %every last column/.style={column type/.add={}{|}},	% style the last column
            %columns/C/.style = {column type/.add={|}{|}}	% style the designated column
        ]{tbl/tab_tan_sin.dat}
    \end{center}        
    \caption[]{Metodo delle tangenti applicato a $sin(x)$ con stima iniziale 3,1 e nmax = 10}
\end{table}

\begin{table}[!htbp]
    \begin{center}
        \pgfplotstabletypeset[
            col sep=tab,
            every head row/.style={before row=\toprule,after row=\midrule},	% style the first row
            every last row/.style={after row=\bottomrule},	% style the last row
            every column/.style={dec sep align,precision=10}
            %columns/.style={sci,sci subscript,sci zerofill,dec sep align}
            %every first column/.style={column type/.add={|}{}},	% style the first column
            %every last column/.style={column type/.add={}{|}},	% style the last column
            %columns/C/.style = {column type/.add={|}{|}}	% style the designated column
        ]{tbl/tab_tan_exp_mx.dat}
    \end{center}        
    \caption[]{Metodo delle tangenti applicato a $e^{e^{-x}}-x$ con stima iniziale 0,5 e nmax = 10}
\end{table}

\begin{figure}[!htbp]
    \centering
    \includestandalone{img/geo_tan}
    \caption{Riepilogo sul metodo delle tangenti applicato a $e^{0.9x - x^2}$}
    \label{fig:geo_tan_sqrt_6}
\end{figure}


\subsubsection{Esempio: algoritmo del reciproco}

Si vuole cercare il reciproco del valore $\nu$ come $\alpha = \frac{1}{\nu}$ con il metodo delle tangenti.

Per prima cosa occorre trasformare il problema con una funzione che si annulla in $\frac{1}{\nu}$.

Scegliamo $$f(x) = \nu - \frac{1}{x}$$ che applicata al reciproco di $\nu$ produce
$f(\frac{1}{\nu}) = \nu - \frac{1}{\frac{1}{\nu}} = \nu - \nu = 0$. 

La derivata prima assume la forma $f'(x) = \frac{1}{x^2}$ e

$$\frac{f(x)}{f'(x)} = \frac{\nu - \frac{1}{x}}{\frac{1}{x^2}} = \nu x^2 - x.$$

L'iterata del metodo delle tangenti è:

$$x_{k+1} = x_k - (\nu x_k^2 - x_k) = 2 x_k - \nu x_k^2  = x_k \cdot (2 - \nu \cdot x_k).$$

Si noti che per calcolare il reciproco di un numero sono sufficienti le operazioni di moltiplicazione e sottrazione.
Per la moltiplicazione occorrono le operazioni primitive di scorrimento e addizione e per la sottrazione quelle di negazione bit a bit e di addizione (basterebbe il solo incremento unitario).

Nota: queste proprietà permettono di realizzare le CPU senza l'operazione di divisione.


\subsubsection{Esempio: metodo di Erone per la radice quadrata}

Si consideri il problema di determinare la radice quadrata di un numero reale $r$.

Se dal punto di vista algebrico possiamo vedere il problema come determinare uno zero della funzione $f(x) = x^2 - r$,
da un punto di vista geometrico possiamo porci il problema di trovare il lato di un quadrato di area $r$.


L'idea di Erone di Alessandria è quella fornire una stima della radice quadrata, che chiamiamo $x_k$ e che
rappresentiamo geometricamente come la base di un rettangolo. Chiamiamo $h_k = \frac{r}{x_k}$ l'altezza del rettangolo
di area $r$ e base $x_k$.
Se $x_k$ e $h_k$ fossero molto vicini tra di loro, allora abbiamo trovato una buona stima per $\sqrt{r}$, altrimenti
occorre migliorare la stima. Per fare ciò consideriamo il valor medio tra $x_k$ e $h_k$.

L'iterata del metodo, ossia la stima migliore, ha equazione:

$$x_{k+1} = \frac{x_k+\frac{r}{x_k}}{2}$$

che può essere riscritta come:

$$x_{k+1} = \frac{x_k+ \frac{r}{x_k}}{2} = \frac{x_k^2 + r}{2x_k}$$

Vogliamo dimostrare che la traccia di esecuzione del metodo di Erone di Alessandria è la stessa del metodo
di Newton applicato alla funzione $f(x) = x^2 - r$ calcolando l'iterata:

$$x_{k+1} = x_k - \frac{f(x_k)}{f'(x_k)} =
            x_k - \frac{x_k^2 - r}{2 x_k} =
            \frac{2 x_k^2 - ( x_k^2 - r)}{2 x_k} =
            \frac{x_k^2 + r}{2 x_k}$$

Le iterate assumono lo stesso valore, come volevasi dimostrare!

Si vedano 

\begin{table}[!htbp]
    \begin{center}
\pgfplotstabletypeset[
	col sep=tab,
    every head row/.style={before row=\toprule,after row=\midrule},	% style the first row
	every last row/.style={after row=\bottomrule},	% style the last row
    every column/.style={dec sep align,precision=10}
	%every first column/.style={column type/.add={|}{}},	% style the first column
	%every last column/.style={column type/.add={}{|}},	% style the last column
	%columns/C/.style = {column type/.add={|}{|}}	% style the designated column
]{tbl/tab_erone_sqrt_6.dat}
\end{center}        
\caption[]{Metodo di Erone di Alessandria per il calcolo di $\sqrt{6}$ con stima iniziale 3 e nmax = 10}
\label{tab:erone_sqrt_6}
\end{table}

\begin{figure}[!htbp]
    \centering
    \includestandalone{img/geo_erone_1}
    \caption{Successione delle soluzioni del metodo di Erone per il calcolo di $\sqrt{6}$ con stima iniziale 3}
    \label{fig:geo_erone_sqrt_6}
\end{figure}

\section{Estensioni per il calcolo degli estremanti relativi}

L'analisi matematica fornisce delle condizioni che mettono in relazione i punti
di massimo e minimo relativo con quelli \textit{stazionari}.

\begin{thm}[Teorema di Fermat sui punti stazionari]
Sia $f :(a, b) \to \mathbb{R}$ una funzione e si supponga che $x_0 \in  (a, b)$
sia un punto di estremo locale di $f$. Se $f$ è derivabile nel punto $x_0$,
allora $f^{\prime}(x_0) = 0$.
\end{thm}

Trovare un estremante di una funzione, se è derivabile con condizioni che
dipendono dal metodo scelto e la derivata è nota in forma simbolica o numerica,
si riconduce al problema di determinare una radice della derivata prima.

Tra i metodi più interessanti si trova quello di Newton.

\subsection{Metodo di Newton per l'ottimizzazione}

Possiamo applicare il metodo di Newton per la ricerca delle radici ad una
funzione $g(x) = f'(x)$ per trovare un punto stazionario. Si richiede che $f$
sia continua e derivabile almeno due volte in un intorno di $x_0$, stima iniziale
del punto estremante.

$$x_{k + 1} = x_k - \frac{g(x_k)}{g'(x_k)} = x_k - \frac{f'(x_k)}{f''(x_k)}$$

\subsubsection[Interpretazione geometrica]{Interpretazione geometrica del metodo di Newton per l'ottimizzazione}

Possiamo sviluppare il metodo in modo geometrico, approssimando $f(x_k)$ con una parabola $p(x) = a x^2 + bx + c$
passante per il punto $\left(x_k, f(x_k)\right)$, con la stessa tangente nel punto $x_k$ e stessa derivata seconda.
Trovati i coefficienti $a, b$ e $c$ della parabola, possiamo approssimare il vertice della funzione $f$
con il vertice della parabola, che sappiamo avere proiezione sull'asse delle ascisse $x_{k+1} = \frac{-b}{2a}$.

Per calcolare $a$ e $b$, gli unici due parametri usati per determinare il vertice, imponiamo le condizioni sulle
derivate prime e seconda.

\systeme{f'(x_k) = 2a x_k + b,
         f''(x_k) = 2a}

Dal sistema abbiamo esplicitato $2a = f''(x_k)$ e dobbiamo ricavare $b$ dalla prima equazione.

$b = f'(x_k) - 2a x_k = f'(x_k) - f''(x_k) x_k$.

L'iterata del metodo geometrico che a partire dal punto $x_k$ approssima l'estremante di $f$
con il vertice di $p$ ha equazione:

$$x_{k+1} = \frac{-b}{2a} = \frac{f''(x_k) x_k - f'(x_k)}{f''(x_k)} = x_k - \frac{f'(x_k)}{f''(x_k)}$$

Per completezza, possiamo determinare tutti i coefficienti di $p$:

\systeme{f(x_k) = a x_k^2 + b x_k + c,
         f'(x_k) = 2a x_k + b,
         f''(x_k) = 2a}

Sostituendo nella prima equazione $a$ e $b$ si ha:

$$c = - \frac{1}{2} f''(x_k) - f'(x_k) x_k + f''(x_k) x_k^2 + f(x_k) = f(x_k)  - f'(x_k) x_k + \frac{1}{2} f''(x_k) x_k^2$$


Questa è l'interpretazione geometrica del metodo di Newton applicato alla derivata prima.

\subsection{Esempio}

Consideriamo la funzione $f(x) = \frac{1}{3} x^3 + 2 x^2 + x + 6$, di cui è agevole
calcolare le derivate prima, $f'(x) = x^2 + 4 x + 1$, e seconda, $f''(x) = 2x + 4$.

Analiticamente è facile trovare gli estremanti in $x_M = -2 \pm \sqrt{3}$.

Numericamente potremmo valutare la funzione in più punti e magari potremmo osservare
che gli estremanti si trovano nell'intervallo $[-5, 2]$ e consideriamo una volta il
punto -3 e l'altra -1 come stime iniziali.

Possiamo riscrivere l'iterata come:

$$x_{k+1} = x_k - \frac{x^2 + 4 x + 1}{2x + 4}$$

I parametri della parabola sono:

\systeme{f(x_k) = a x_k^2 + b x_k + c,
         f'(x_k) = 2a x_k + b,
         f''(x_k) = 2a}

\systeme{c = \frac{1}{3} x_k^3 + 6,
b = 1 - x_k^2 ,
a = x_k + 2}

\begin{table}[!htbp]
    \begin{center}
\pgfplotstabletypeset[
	col sep=tab,
    every head row/.style={before row=\toprule,after row=\midrule},	% style the first row
	every last row/.style={after row=\bottomrule},	% style the last row
    every column/.style={dec sep align,precision=10}
]{tbl/tab_par_es1.dat}
\end{center}        
\caption[]{Metodo delle parabole con nmax = 5. $x_0 = -3$}
\label{tbl:par_es1}
\end{table}

\begin{table}[!htbp]
    \begin{center}
\pgfplotstabletypeset[
	col sep=tab,
    every head row/.style={before row=\toprule,after row=\midrule},	% style the first row
	every last row/.style={after row=\bottomrule},	% style the last row
    every column/.style={dec sep align,precision=10}
]{tbl/tab_par_es2.dat}
\end{center}        
\caption[]{Metodo delle parabole con nmax = 5. $x_0 = -1$}
\label{tbl:par_es2}
\end{table}

% \begin{figure}[!htbp]
%     \centering
%     \includestandalone{img/iter_cor_sqrt_6}
%     \caption{Successione delle soluzioni del metodo delle secanti applicato a $x^2 -6$ nell'intervallo $[0, 6]$}
%     \label{fig:par_es1}
% \end{figure}
