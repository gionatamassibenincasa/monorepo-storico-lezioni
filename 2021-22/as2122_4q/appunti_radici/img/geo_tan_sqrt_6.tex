\documentclass[tikz]{standalone}
\usepackage{pgfplots}
\pgfplotsset{compat=1.18}
\begin{document}
\begin{tikzpicture}
\begin{axis}[
    axis lines=middle,
    % axis equal image,
    xlabel=$x$,
    ylabel=$y$,
    xmin=2,xmax=4,
    ymin=-5,ymax=10,
    xtick={2, 0 , 2, 4},
    ytick={-5, 0, 5, 10},
    legend style={at={(axis cs:1,1)},anchor=south west}
]
\addplot[domain=2:4,samples=100,color=red]{x^2-6};
\addlegendentry{$f(x)=x^2-6$}
\xdef\xold{3}
\xdef\yold{3}
\foreach \k in {0,...,2}{
    \pgfmathsetmacro{\xnew}{\xold - (pow(\xold, 2)-6)/(2*\xold)}
    \pgfmathsetmacro{\ynew}{pow(\xnew, 2)-6}
    % il punto sulla funzione
    \edef\punto{\noexpand\addplot[
        only marks,
        mark=circle,
        mark size=2pt,
        point meta=explicit symbolic,
        nodes near coords,
        ] coordinates {%
            (\xold,\yold) [$x_{\k}$]%
        };
    }
    \punto
      
    \ifodd\k
        % la tangente
        \addplot[domain=2:4,color=green]{2*\xold*(x-\xold)+\yold};
        % il segmento verticale
        \addplot[mark=none,color=green] coordinates {(\xold,\yold) (\xold,0)};
    \else
        % il punto sulla funzione
        \addplot[mark=o,color=blue] coordinates {(\xold,\yold) (\xnew,\ynew)};
        % la tangente
        \addplot[domain=2:4,color=blue]{2*\xold*(x-\xold)+\yold};
        % il segmento verticale
        \addplot[mark=none,color=blue] coordinates {(\xold,\yold) (\xold,0)};
    \fi
%    \addlegendentry{$k=\k$}
    \global\let\xold\xnew
    \global\let\yold\ynew
}
\end{axis}
\end{tikzpicture}
\end{document}