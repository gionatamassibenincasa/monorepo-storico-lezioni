% https://tex.stackexchange.com/questions/551160/plot-that-demonstrate-newtons-method
%\documentclass[]{article}
\documentclass[margin=5pt, varwidth]{standalone}
\usepackage[utf8]{inputenc}
\usepackage{amsmath}
\usepackage{pgfplots}
\usepackage{pgfplotstable}
\pgfplotsset{compat=newest}

\begin{document}
% Input 1/2 =====
\newcommand\fxshow{e^{0.9x}-x^2}
\pgfmathsetlengthmacro\mywidth{8.9cm}

\tikzset{trig format=rad, 
declare function={
% Input 2/2 =====
f(\x)=exp(0.9*\x) -\x*\x;  
xStart=2.6;
Steps=4;
% Calc ====
xNew(\x)=\x-f(\x)/df(\x);
dx=0.01;      
df(\x)=( f(\x+dx) -f(\x) )/dx;
},}

% Start row
\pgfmathsetmacro\xStart{xStart}
\pgfmathsetmacro\fxnStart{f(xStart)}
\pgfmathsetmacro\dfxnStart{df(xStart)}
\pgfmathsetmacro\xNewStart{xNew(xStart)}
\pgfplotstableread[header=false, col sep=comma,
]{
0, \xStart, \fxnStart, \dfxnStart,  \xNewStart
}\newtontable

% Further rows
\pgfmathsetmacro\Steps{Steps}
\pgfplotsforeachungrouped \n in {1,...,\Steps} {%%
\ifnum\n=1 \pgfplotstablegetelem{0}{[index]4}\of\newtontable \else
\pgfplotstablegetelem{0}{[index]4}\of\nextrow \fi
\pgfmathsetmacro\xOld{\pgfplotsretval}
%
\pgfmathsetmacro\fxn{f(\xOld)}
\pgfmathsetmacro\dfxn{df(\xOld)}
\pgfmathsetmacro\xNew{xNew(\xOld)}
%
\edef\createnextrow{
\noexpand\pgfplotstableread[
col sep=comma,      row sep=crcr, 
]{
\n,   \xOld,   \fxn, \dfxn, \xNew \noexpand\\
}\noexpand\nextrow
}\createnextrow
%
% Concatenate in loop
\pgfplotstablevertcat{\temprow}{\nextrow}
%\n \pgfplotstabletypeset{\temprow} \\ % Show for test
}%%
% Concatenate with startrow
\pgfplotstablevertcat{\newtontable}{\temprow}

% Output =============================
\pgfmathsetmacro\dx{dx}

\newsavebox{\ExampleText}
\savebox\ExampleText{% ======================
\begin{minipage}{\mywidth}
% Title =======
$f(x) = \fxshow   \\[1em]
f'(x)\approx \dfrac{f(x+\Delta x)-f(x)}{\Delta x},~~\Delta x=\dx \\[0.5em]
t_k(x) = f'(x_k)\cdot (x-x_k)+f(x_k) \\[0.5em]
x_0=\xStart,~~    x_{k+1}=x_k-\dfrac{f(x_k)}{f'(x_k)}    $  \\[0.5em]
%Table =======
\pgfplotstabletypeset[column type=r, 
% Show integers as intgers and general number format:
every column/.style={postproc cell content/.style={
@cell content=\pgfmathifisint{##1}
        {\pgfmathprintnumber[precision=0]{##1}}  
        {\pgfmathprintnumber[fixed,  fixed zerofill,  precision=5]{##1}}  
}}, 
%font=\footnotesize, 
display columns/0/.style={column name=$n$},
display columns/1/.style={column name=$x_k$},
display columns/2/.style={column name=$f(x_k)$},
display columns/3/.style={column name=$f'(x_k)$},
display columns/4/.style={column name=$x_{k+1}$},
every head row/.style={after row=\hline, before row=\hline},
every last row/.style={after row=\hline},
]{\newtontable} \\[0.5em]
%
\xdef\xRes{\xNew}
\pgfmathparse{f(\xRes)}
\xdef\yRes{\pgfmathresult}
{$\Rightarrow~ \boldsymbol{ x  \approx\xNew}$  }
\end{minipage}}%========================
%\usebox{\ExampleText} % Show for test

\begin{tikzpicture}[
font=\footnotesize, 
]
% Curve =============================
\begin{axis}[local bounding box=Curve,
%width=\mywidth, 
title={\usebox{\ExampleText}},
title style={align=left, anchor=south west, 
draw=none, text width=\mywidth, 
at={(rel axis cs:0,1)},   name=Example, 
},
trig format=rad, 
axis lines = center,
xlabel=$x$,
ylabel=$y$,
axis line style = {-latex},
xlabel style={anchor=north},
ylabel style={anchor=east},
xmin=-3,      xmax=3,
%ymin=-0.5,     ymax=3.7,
%xtick={-1,-0.6,...,1},
%minor ytick={-0.5,0,...,3.5},
%legend pos=outer north east,
legend style={at={(0.0,-0.05)},anchor=north west},
legend cell align=left,
enlarge y limits=upper,
enlarge x limits,
clip=false, 
]
% Curve
\addplot[thick, domain=-1.5:3, blue]{f(x)}; 
\addlegendentry{$f(x)=\fxshow$}
% Tangents
\foreach \row in {0,...,\Steps}{%%
\pgfplotstablegetelem{\row}{0}\of\newtontable
\xdef\Index{\pgfplotsretval}
\pgfplotstablegetelem{\row}{1}\of\newtontable
\xdef\xS{\pgfplotsretval}
\pgfmathsetmacro\xSi{-\xS}
\pgfmathsetmacro\xSshow{\xSi<0 ? \xSi : "+\xSi"}
%
\pgfplotstablegetelem{\row}{2}\of\newtontable
\xdef\yS{\pgfplotsretval}
\pgfmathsetmacro\ySshow{\yS<0 ? \yS : "+\yS"}
%
\pgfplotstablegetelem{\row}{3}\of\newtontable
\xdef\dyS{\pgfplotsretval}
% 
\pgfmathsetmacro\vR{0.4+1/\dyS}
\pgfmathsetmacro\vL{1.1+1/\dyS}
\pgfmathsetmacro\Pos{\row==3 || \row==999 ? -0.05 : 1.05}

\edef\nextplot{
\noexpand\addplot[red, domain=\xS-\vL:\xS+\vR, forget plot]{\dyS*(x-\xS)+\yS} node[pos=\Pos]{$t_\Index$}; 
\noexpand\addplot[red, mark=*, mark size=1.5pt, mark options={fill=white, draw=black}] coordinates{(\xS,\yS) };
\noexpand\addlegendentry[]{$t_\Index(x)=\dyS\cdot (x \xSshow) \ySshow$}
\noexpand\addplot[densely dashed, forget plot] coordinates{(\xS,\yS) (\xS,0)} node[below]{$x_\Index$};
}\nextplot
}%

% Zero of Curve
\addplot[mark=*, mark size=1.75pt, forget plot] coordinates{(\xRes,\yRes)};
\end{axis}
\end{tikzpicture}
\end{document}

Descrizione: Una funzione con andamento esponenziale e quattro tangenti, indicazione del calcolo numerico della tangente usando un incremento finito
Argomento sotteso: Metodo di Newton (delle tangenti) per la determinazione degli zeri di funzione

Immagine che mostra una parabola e l'interpretazione grafica di un'iterazione generica del metodo (algoritmo) di Newton.

Informatica:
Derivare l'iterata x_{k+i} del metodo di Newton nel caso generale.
E' importante notare che l'algoritmo può convergere, oscillare o divergere.
Per molte funzioni, il metodo di Newton, se converge lo fa più velocemente del metodo di bisezione.
Dire che il problema della derivazione in Informatica è un problema difficile mentre in matematica, a livello
simbolico, è un problema semplice. L'opposto accadde per l'integrazione.

Possibili collegamenti

Matematica: Funzione esponenziale, derivata di un'esponenziale, integrale di un'esponenziale.

Fisica: fenomeni con andamento esponenziale, come i decadimenti...

Italiano: Decadentismo (per assonanza con decadimento).

Storia: Seconda Guerra Mondiale/primo dopoguerra

Filosofia: Russell (pacifismo, II guerra mondiale)??? Marx (ideologie e totalitarismi del XX secolo)???

Inglese: Stream of Consciousness, Joyce

Scienze: qualunque cosa che abbia a che fare con un'esponenziale