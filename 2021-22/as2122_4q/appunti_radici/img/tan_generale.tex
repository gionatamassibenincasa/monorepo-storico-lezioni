% https://tex.stackexchange.com/questions/551160/plot-that-demonstrate-newtons-method
\documentclass[border=0.1cm]{standalone}
\usepackage{tikz}
\usetikzlibrary{intersections,calc}

\begin{document}
\begin{tikzpicture}[thick,yscale=0.8]

% Axes
\draw[-latex,name path=xaxis] (-1,0) -- (12,0) node[above]{\large $x$};
\draw[-latex] (0,-2) -- (0,8)node[right]{\large $y$};;

% Function plot
\draw[ultra thick, orange,name path=function]  plot[smooth,domain=1:9.5] (\x, {0.1*\x^2-1.5}) node[left]{$f(x)$};

% plot tangent line
\node[violet,right=0.2cm] at (8,4.9) {\large tangente};
\draw[gray,thin,dotted] (8,0) -- (8,4.9) node[circle,fill,inner sep=2pt]{};
\draw[dashed, violet,name path=Tfunction]  plot[smooth,domain=4.25:9.5] (\x, {1.6*\x-7.9});

% x-axis labels
\draw (8,0.1) -- (8,-0.1) node[below] {$x_{k}$};
\draw [name intersections={of=Tfunction and xaxis}] ($(intersection-1)+(0,0.1)$) -- ++(0,-0.2) node[below,fill=white] {$x_{k+1}$} ;

\end{tikzpicture}
\end{document}

Descrizione: Una parabola e una sua retta con l'indicazione di un punto sull'asse delle ordinate e l'intersezione della tangente con l'asse delle ascisse
Argomento sotteso: Metodo di Newton (delle tangenti) per la determinazione degli zeri di funzione

Immagine che mostra una parabola e l'interpretazione grafica di un'iterazione generica del metodo (algoritmo) di Newton.

Informatica:
Derivare l'iterata x_{k+i} del metodo di Newton nel caso di una parabola, o se si preferisce nel caso generale.
Nel caso della parabola, si possono evitare le derivate in forma esplicita e basta la geometria del terzo anno.
Mostrare che il problema di trovare il punto fisso della funzione x^2-r è equivalente a calcolare la radice quadrata
e che l'iterata è la stessa del metodo di Erone.
E' importante notare che l'algoritmo, nel caso di una parabola, converge sempre.

Possibili collegamenti

Matematica: Equazione della parabola; tangente ad una parabola; tangente.

Fisica: velocità istantanea, accelerazione istantanea, moto parabolico, fisica classica (matematica del continuo) in contrapposizione alla fisica quantistica (matematica del discreto)...

Italiano: (In analogia alla possibilità di conoscere in modo deterministico l'evoluzione di un fenomeno) Positivismo letterario, naturalismo, verismo in contrapposizione a romanticismo.

Storia: Dall'Ottocenteto alla Belle Epoque o alla Prima Guerra Mondiale, oppure rivoluzione industriale

Filosofia:  Comte, positivismo oppure Illuminismo

Inglese: Charles Dickens

Scienze: qualunque cosa che abbia a che fare con una parabola o una retta