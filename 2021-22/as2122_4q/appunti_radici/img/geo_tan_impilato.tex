% https://tex.stackexchange.com/questions/551160/plot-that-demonstrate-newtons-method
%\documentclass[]{article}
\documentclass[margin=5pt, varwidth]{standalone}
\usepackage[utf8]{inputenc}
\usepackage{amsmath}
\usepackage{pgfplots}
\usetikzlibrary{pgfplots.groupplots}

\pgfplotsset{compat=newest, width=10cm}

\begin{document}
% Input 1/2 =====
\newcommand\fxshow{e^{0.9x}-x^2}

\tikzset{trig format=rad, 
declare function={
% Input 2/2 =====
f(\x)=exp(0.9*\x) -\x*\x;  
xStart=2.6;
Steps=4;
% Calc ====
xNew(\x)=\x-f(\x)/df(\x);
df(\x)=0.9*exp(0.9*\x) - 2*\x;
},}

% Starting point
\pgfmathsetmacro\xStart{xStart}
\pgfmathsetmacro\fxnStart{f(xStart)}
\pgfmathsetmacro\dfxnStart{df(xStart)}
\pgfmathsetmacro\xNewStart{xNew(xStart)}
%
\pgfmathsetmacro\fxn{f(\xStart)}
\pgfmathsetmacro\dfxn{df(\xStart)}
\pgfmathsetmacro\xNew{xNew(\xStart)}

\begin{tikzpicture}[font=\footnotesize,]
    \begin{axis}[xmin=-1.5,xmax=3.5,axis lines=middle,xlabel=$x$,ylabel=$y$]
        \addplot[domain=-1.5:3,samples=21] {exp(0.9*x) - pow(x,2)};
    \end{axis}
\end{tikzpicture}
\end{document}