\subsection{Metodo delle corde}

Si considerino due punti $a=x_0$ e $b=x_1$ tali da soddisfare le ipotesi del teorema di Bolzano.
È possibile costruire una successione che per ogni $k \geq 0$  il punto $x_{k+1}$ sia lo zero della retta passante per il punto
($x_{k}$, $f(x_{k})$) e di coefficiente angolare

$$\displaystyle m_k = \frac{f(a)  - f(x_k)}{a - x_k}.$$

L'iterata ha equazione:

$$x_{k+1} = x_k - f(x_k) \frac{a  - x_k)}{f(a) - f(x_k)}.$$

\subsubsection{Convergenza}

Il metodo non necessariamente converge.
Può oscillare (vedi tab.~\ref{tbl:tab_cor_osc}) o divergere.


\subsubsection{Codifica in JavaScript}

Si veda il listato \ref{lst:corde}.

\begin{lstfloat}
    \lstinputlisting{lst/corde.js}
    \caption{Descrizione in JavaScript del metodo delle corde}
    \label{lst:corde}
\end{lstfloat}

\subsubsection{Esempi}

Si vedano le tabb.~\ref{tbl:tab_cor_sqrt_6}, \ref{tbl:tab_cor_sin} e \ref{tbl:tab_cor_exp_mx} e la fig. \ref{fig:cor_sqrt_6}.

\begin{table}
    \begin{center}
\pgfplotstabletypeset[
	col sep=tab,
    every head row/.style={before row=\toprule,after row=\midrule},	% style the first row
	every last row/.style={after row=\bottomrule},	% style the last row
    every column/.style={dec sep align,precision=10}
]{tbl/tab_bis_sqrt_6.dat}
\end{center}        
\caption[]{Metodo delle corde applicato a $x^2 -6$ nell'intervallo $[0, 6]$ con nmax = 10}
\label{tbl:tab_cor_sqrt_6}
\end{table}

\begin{figure}[ht]
    \centering
    \includestandalone{img/iter_cor_sqrt_6}
    \caption{Successione delle soluzioni del metodo delle corde applicato a $x^2 -6$ nell'intervallo $[0, 6]$}
    \label{fig:cor_sqrt_6}
\end{figure}

\begin{table}
    \begin{center}
        \pgfplotstabletypeset[
            col sep=tab,
            every head row/.style={before row=\toprule,after row=\midrule},	% style the first row
            every last row/.style={after row=\bottomrule},	% style the last row
            every column/.style={dec sep align,precision=10}
        ]{tbl/tab_cor_sin.dat}
    \end{center}        
    \caption[]{Metodo delle corde applicato a $sin(x)$ nell'intervallo $[3, 3.2]$ con nmax = 10}
    \label{tbl:tab_cor_sin}
\end{table}

\begin{table}
    \begin{center}
        \pgfplotstabletypeset[
            col sep=tab,
            every head row/.style={before row=\toprule,after row=\midrule},	% style the first row
            every last row/.style={after row=\bottomrule},	% style the last row
            every column/.style={dec sep align,precision=10}
            %columns/.style={sci,sci subscript,sci zerofill,dec sep align}
            %every first column/.style={column type/.add={|}{}},	% style the first column
            %every last column/.style={column type/.add={}{|}},	% style the last column
            %columns/C/.style = {column type/.add={|}{|}}	% style the designated column
        ]{tbl/tab_cor_exp_mx.dat}
    \end{center}        
    \caption[]{Metodo delle corde applicato a $e^{e^{-x}}-x$ nell'intervallo $[0, 1]$ con nmax = 10}
    \label{tbl:tab_cor_exp_mx}
\end{table}

\begin{table}
    \begin{center}
\pgfplotstabletypeset[
	col sep=tab,
    every head row/.style={before row=\toprule,after row=\midrule},	% style the first row
	every last row/.style={after row=\bottomrule},	% style the last row
    every column/.style={dec sep align,precision=10}
]{tbl/tab_cor_osc.dat}
\end{center}        
\caption[]{Metodo delle corde applicato a $x^2 -1$ nell'intervallo $[0, 2]$ con nmax = 10}
\label{tbl:tab_cor_osc}
\end{table}
